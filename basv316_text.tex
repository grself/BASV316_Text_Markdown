\documentclass[]{book}
\usepackage{lmodern}
\usepackage{amssymb,amsmath}
\usepackage{ifxetex,ifluatex}
\usepackage{fixltx2e} % provides \textsubscript
\ifnum 0\ifxetex 1\fi\ifluatex 1\fi=0 % if pdftex
  \usepackage[T1]{fontenc}
  \usepackage[utf8]{inputenc}
\else % if luatex or xelatex
  \ifxetex
    \usepackage{mathspec}
  \else
    \usepackage{fontspec}
  \fi
  \defaultfontfeatures{Ligatures=TeX,Scale=MatchLowercase}
\fi
% use upquote if available, for straight quotes in verbatim environments
\IfFileExists{upquote.sty}{\usepackage{upquote}}{}
% use microtype if available
\IfFileExists{microtype.sty}{%
\usepackage{microtype}
\UseMicrotypeSet[protrusion]{basicmath} % disable protrusion for tt fonts
}{}
\usepackage[margin=1in]{geometry}
\usepackage{hyperref}
\hypersetup{unicode=true,
            pdftitle={Introductory Methods of Analysis},
            pdfauthor={George Self},
            pdfborder={0 0 0},
            breaklinks=true}
\urlstyle{same}  % don't use monospace font for urls
\usepackage{natbib}
\bibliographystyle{apalike}
\usepackage{longtable,booktabs}
\usepackage{graphicx,grffile}
\makeatletter
\def\maxwidth{\ifdim\Gin@nat@width>\linewidth\linewidth\else\Gin@nat@width\fi}
\def\maxheight{\ifdim\Gin@nat@height>\textheight\textheight\else\Gin@nat@height\fi}
\makeatother
% Scale images if necessary, so that they will not overflow the page
% margins by default, and it is still possible to overwrite the defaults
% using explicit options in \includegraphics[width, height, ...]{}
\setkeys{Gin}{width=\maxwidth,height=\maxheight,keepaspectratio}
\IfFileExists{parskip.sty}{%
\usepackage{parskip}
}{% else
\setlength{\parindent}{0pt}
\setlength{\parskip}{6pt plus 2pt minus 1pt}
}
\setlength{\emergencystretch}{3em}  % prevent overfull lines
\providecommand{\tightlist}{%
  \setlength{\itemsep}{0pt}\setlength{\parskip}{0pt}}
\setcounter{secnumdepth}{5}
% Redefines (sub)paragraphs to behave more like sections
\ifx\paragraph\undefined\else
\let\oldparagraph\paragraph
\renewcommand{\paragraph}[1]{\oldparagraph{#1}\mbox{}}
\fi
\ifx\subparagraph\undefined\else
\let\oldsubparagraph\subparagraph
\renewcommand{\subparagraph}[1]{\oldsubparagraph{#1}\mbox{}}
\fi

%%% Use protect on footnotes to avoid problems with footnotes in titles
\let\rmarkdownfootnote\footnote%
\def\footnote{\protect\rmarkdownfootnote}

%%% Change title format to be more compact
\usepackage{titling}

% Create subtitle command for use in maketitle
\newcommand{\subtitle}[1]{
  \posttitle{
    \begin{center}\large#1\end{center}
    }
}

\setlength{\droptitle}{-2em}

  \title{Introductory Methods of Analysis}
    \pretitle{\vspace{\droptitle}\centering\huge}
  \posttitle{\par}
    \author{George Self}
    \preauthor{\centering\large\emph}
  \postauthor{\par}
      \predate{\centering\large\emph}
  \postdate{\par}
    \date{2018-10-23}

\usepackage{booktabs}
\usepackage{amsthm}
\makeatletter
\def\thm@space@setup{%
  \thm@preskip=8pt plus 2pt minus 4pt
  \thm@postskip=\thm@preskip
}
\makeatother

\usepackage{amsthm}
\newtheorem{theorem}{Theorem}[chapter]
\newtheorem{lemma}{Lemma}[chapter]
\theoremstyle{definition}
\newtheorem{definition}{Definition}[chapter]
\newtheorem{corollary}{Corollary}[chapter]
\newtheorem{proposition}{Proposition}[chapter]
\theoremstyle{definition}
\newtheorem{example}{Example}[chapter]
\theoremstyle{definition}
\newtheorem{exercise}{Exercise}[chapter]
\theoremstyle{remark}
\newtheorem*{remark}{Remark}
\newtheorem*{solution}{Solution}
\begin{document}
\maketitle

{
\setcounter{tocdepth}{1}
\tableofcontents
}
\hypertarget{preface}{%
\chapter*{Preface}\label{preface}}
\addcontentsline{toc}{chapter}{Preface}

I have taught BASV 316, \emph{Introductory Methods of Analysis}, online
for the University of Arizona in Sierra Vista since 2010 and enjoy
working with students on research methods. From the start, I wanted
students to work with statistics as part of our studies and carry out
the types of calculations that are discussed in the text. As I evaluated
statistical software I had three criteria:

\begin{itemize}
\item
  \textbf{Open Educational Resource (OER)}. It is important to me that
  students use software that is available free of charge and is
  supported by the entire web community.
\item
  \textbf{Platform}. While most of my students use a Windows-based
  system, some use Macintosh and it was important to me to use software
  that is available for all of those platforms. As a bonus, most OER
  software is also available for the Linux system, though I am not aware
  of any of my students who are using Linux. Finally, I occasionally
  have students who are not able to load software on their personal
  computers (think: \emph{Chromebook}) so I needed an online capability.
\item
  \textbf{Longevity}. I wanted a system that could be used in other
  college classes or in a business setting after graduation. That way,
  any time a student spends learning the software in my class will be an
  investment that can yield results for many years.
\end{itemize}

\emph{R} (just a single letter, \emph{R}) met those objectives and that
is the software I chose to use. This manual started as a series of six
lab exercises using \emph{R} but has grown over the years to the ten
topics covered in this edition. Moreover, \emph{R} is a recognized
standard for statistical analysis and could be easily used for even
peer-reviewed published papers. It is my hope that students will find
the labs instructive and they will then be able to use \emph{R} for
other classes.

This lab manual is written with \emph{Bookdown} tools in \emph{RStudio}.
It is published under a \emph{Creative Commons 0 Universal} license,
essentially ``public domain,'' (see {[}Creative Commons License{]}) with
a goal that other instructors can modify and use it to meet their own
needs. The source can be found at \href{https://goo.gl/6dZqCK}{GITHUB}
and I always welcome comments. Finally, it was written with base R (R
Core Team (2017). R: A language and environment for statistical
computing. \href{https://www.R-project.org/}{R Foundation for
Statistical Computing, Vienna, Austria.}).

--George Self

\hypertarget{introduction}{%
\chapter{Introduction}\label{introduction}}

\hypertarget{objectives}{%
\section{Objectives}\label{objectives}}

\begin{enumerate}
\def\labelenumi{\arabic{enumi}.}
\tightlist
\item
  Identify the various sources of knowledge
\item
  Define ``science''
\item
  Describe the scientific method and relate that to business research
\item
  Identify the three types of science research (exploratory,
  descriptive, and explanatory)
\end{enumerate}

\hypertarget{knowing}{%
\section{Knowing}\label{knowing}}

In general, people want to know about things. Most people are curious
about the world around them but business owners are interested in
specifically how people can be persuaded to make a purchase.
Understanding how one person can walk past a candy store without even
the slightest thought about going inside while another cannot seem to
walk in the same block without stopping in for a treat is valuable
information for the owner of the candy store. In general, business
owners are eager to know about people and what drives their behavior.

The goal of this book is to teach students how research can be used to
help business owners make good decisions. More specifically, the book
examines the ways that researchers come to understand the impetus that
drives purchases. The research methods considered in this book are a
systematic process of inquiry designed to learn something of value about
a business problem. Before considering research methods, though, it is
useful to contemplate other sources of knowledge.

\hypertarget{different-sources-of-knowledge}{%
\subsection{Different Sources of
Knowledge}\label{different-sources-of-knowledge}}

As an introduction to the field of research, it is useful to briefly
consider common sources of knowledge.

\begin{enumerate}
\def\labelenumi{\arabic{enumi}.}
\item
  Assumptions. Many people assume that children without siblings are
  rather spoiled and unpleasant. In fact, many people believe that the
  social skills of only children will not be as well developed as those
  of people who were reared with siblings. However, sociological
  research shows that children who grow up without siblings are no worse
  off than those with siblings when it comes to developing good social
  skills \footnote{Bobbitt-Zeher, Donna, and Douglas B. Downey. ``Number
    of siblings and friendship nominations among adolescents.'' Journal
    of Family Issues 34.9 (2013): 1175-1193.}. Researchers consider
  precisely these types of assumptions that ``everyone knows'' when
  investigating their worlds. Sometimes the assumptions are correct and
  other times not so much.
\item
  Direct Experience. One source of knowledge is direct experience. Mark
  Twain observed that ``\ldots{} the cat that sits down on a hot
  stove-lid \ldots{} will never sit down on a hot stove-lid
  again\ldots{}'' \footnote{Twain, Mark. Following the equator.
    Trajectory Inc, 2014. Found at
    (\url{https://fada.birzeit.edu/jspui/bitstream/20.500.11889/4989/1/following_the_equator__a_journey_around_the_world.pdf})}
  Direct experience may be a source of accurate information, but only
  for those who experience it. The problem is that the observation is
  not deliberate or formal; rather, it comes as an accidental by-product
  of life. Even worse, the lesson learned may be wrong. Without a
  systematic process for observing and evaluating those observations any
  conclusions drawn are suspect.
\item
  Tradition. Another source of knowledge is tradition. There is an urban
  legend about a woman who for years used to cut both ends off of a ham
  before putting it in the oven \footnote{See Snopes:
    (\url{http://www.snopes.com/weddings/newlywed/secret.asp})}. She
  baked ham that way because that was the way her mother did it, so
  clearly that was the way it was supposed to be done. Her mother was
  the authority, after all. After years of tossing cuts of perfectly
  good ham into the trash, however, she learned that the only reason her
  mother ever cut the ends off ham before cooking it was that her baking
  pan was not large enough to accommodate the ham without trimming it.
  Tradition may or may not be a good source of knowledge.
\item
  Authority. Many people rely on the government, teachers, and other
  authority figures to dispense knowledge. Unfortunately, authority
  figures may or may not be a source of accurate knowledge.
\item
  \textbf{Observation}. People rely on their own informal observations
  of their worlds. Occasionally, someone will decide to ``investigate''"
  something, perhaps an odd sound, and their observations will become
  more selective. Unfortunately, these types of observations are not
  systematic and may easily lead to incorrect conclusions.
\item
  Generalization. Often a broad pattern is observed and people draw a
  conclusion that the pattern is true for all instances. This can be the
  source of prejudice where the actions of a few bad actors may bias
  peoples' knowledge of the whole.
\end{enumerate}

While there are many ways that people come to know what they know, some
of those ways are more reliable than others. The goal of formal research
is to ferret out an accurate answer to the questions people have --- to
provide a reliable source of knowledge.

\hypertarget{what-is-science}{%
\subsection{What is science?}\label{what-is-science}}

Most research methods used for business and marketing are based on
methods used in the various social sciences and this section of the book
describes how that scientific research is conducted.

Many students assume that ``science'' is a craft practiced by highly
educated experts wearing white lab coats and pouring boiling liquids
into test tubes. Unfortunately, that is not an accurate definition of
``science.'' Etymologically, the word ``science'' is derived from the
Latin word \emph{scientia}, which means knowledge. ``Science,'' then, is
a systematic and organized body of knowledge acquired by using a
specific, rigorous method in any field of inquiry. The sciences can be
grouped into two broad categories: natural and social. Natural science
is the science of naturally occurring objects or phenomena, such as
light, objects, matter, earth, celestial bodies, or the human body.
Natural sciences are further classified into the physical sciences,
earth sciences, life sciences, and others. In contrast, social science
is the science of people or collections of people, such as groups,
firms, societies, or economies, and their individual or collective
behaviors. Social sciences can be classified into disciplines such as
psychology (the science of human behaviors), sociology (the science of
social groups), and economics (the science of markets and economies).

Sciences are also classified by their purpose. Basic sciences, also
called pure sciences, are those that explain the most basic objects and
forces, relationships between them, and laws governing them. Examples
include physics, mathematics, and biology. Applied sciences, also called
practical sciences, are sciences that apply scientific knowledge from
basic sciences to a physical environment. For instance, engineering is
an applied science that applies the laws of physics and chemistry to
practical applications such as building stronger bridges or fuel
efficient combustion engines, while medicine is an applied science that
applies the laws of biology to relieving human ailments.

Scientific knowledge is a generalized body of laws and theories acquired
using the scientific method to explain a phenomenon or behavior of
interest. Closely related to laws and theories are hypotheses.

\begin{itemize}
\item
  Laws are observed patterns of phenomena or behaviors and are based on
  repeated experimental observations. They are generalized rules that
  explain observations and are, typically, theories that have been
  repeatedly tested and believed to be true. As an example, the
  Newtonian Laws of Motion describe what happens when an object is in a
  state of rest or motion (Newton's First Law), what force is needed to
  move a stationary object or stop a moving object (Newton's Second
  Law), and what happens when two objects collide (Newton's Third Law).
  Collectively, the three laws constitute the basis of classical
  mechanics --- a theory of moving objects.
\item
  Theories are systematic explanations of underlying phenomenon or
  behavior. Theories are typically based on hypotheses that have been
  tested and found to be true, but the testing has been incomplete or
  not rigorous enough to classify the theory as a law. It is important
  to note that theories are not ``wild guesses'' but are, instead, the
  result of experimental observations that found to be true in the
  instances tested. It is also important to note that theories can be
  falsified, that is, there are ways to prove that the theory is not
  true. As examples, the theory of optics explains the properties of
  light and how it behaves in different media, electromagnetic theory
  explains the properties of electricity and how to generate it, quantum
  mechanics explains the properties of subatomic particles, and
  thermodynamics explains the properties of energy and mechanical work.
\item
  Hypotheses are a well-guessed explanation of some phenomena or a
  prediction about what will happen in the future. Hypotheses are
  generally the beginning of an investigation that will either support
  or reject the hypotheses. As an example, a researcher may hypothesize
  that products in red boxes sell better than products in blue boxes. To
  test the hypothesis, an experiment can be set up where the same
  product is sold in two identical boxes, except that one box is red and
  the other blue.
\end{itemize}

The pure science of economics and its applied science of business
includes a body of both laws and theories. For example:

\begin{itemize}
\item
  Law of Supply and Demand. While this is often described as a
  \emph{model} it is also usually categorized as a law since it has been
  shown to be true in repeated observations. This law basically states
  that there is a relationship between a product's demand and its
  supply.
\item
  Law of Diminishing Returns. This law states that at some point
  increasing a single production factor will yield less profit-per-unit
  produced. In other words, the return on the investment is not worth
  the cost.
\item
  The 2009 Nobel Prize for economics was for the theory that groups work
  together to manage common resources, like water, by using collective
  property rights.
\item
  The theory of marginalism attempts to explain the discrepancy in the
  value of goods by looking at their secondary, or marginal, utility.
  The price of diamonds is greater than water because of a marginal
  ``satisfaction'' of owning diamonds when compared to water, even
  though water is far more utilitarian.
\end{itemize}

The goal of scientific research is to discover laws and postulate
theories that can explain natural or social phenomena, or in other
words, build scientific knowledge. It is important to understand that
this knowledge may be imperfect or even quite far from the truth. It is
important to understand that theories, upon which scientific knowledge
is based, are explanations of a particular phenomenon and some tend to
fit the observations better than others. The progress of science is
marked by progression over time from poorer theories to better theories
through enhanced observations using more accurate instruments and more
informed logical reasoning.

Scientific laws or theories are derived through a process of logic and
evidence. Logic (theory) and evidence (observations) are the two, and
only two, pillars upon which scientific knowledge is based. In science,
theories and observations are interrelated and one cannot exist without
the other. Theories provide meaning and significance to what we observe
and observations help validate or refine existing theory or construct
new theory. Any other means of knowledge acquisition, such as faith or
authority, cannot be considered science.

\hypertarget{scientific-research}{%
\subsection{Scientific Research}\label{scientific-research}}

Scientific research moves easily between theory and observations, each
reinforcing the other. Theory drives the research of some phenomenon but
observations made by the research further refine the underlying theory.
Relying solely on observations for making inferences while ignoring
theory is not scientific research, it is simple observation. The
application of theories and observations lead to two primary types of
scientific research: theoretical and empirical. Theoretical research is
concerned with developing abstract concepts about natural or social
phenomena while empirical research is concerned with testing theoretical
concepts to see how well they reflect reality in our observations.

Depending on a researcher's training and interest, scientific inquiry
may take one of two forms: \emph{inductive research} or \emph{deductive
research}. The goal of inductive research is to infer theoretical
concepts and patterns from observed data. In contrast, the goal of
deductive research, is to test theory using empirical data. Hence,
inductive research is sometimes called theory-building research while
deductive research is called theory-testing research. Note here that the
goal of theory-testing is not just to test a theory, but to refine,
improve, and extend it.

The above figure illustrates the complementary nature of inductive and
deductive research; they are two halves of a research cycle that
constantly iterates. It is important to understand that theory-building
(inductive research) and theory testing (deductive research) are both
critical for the advancement of science and they are covered more
thoroughly in Chapter 2. Elegant theories are not valuable if they do
not match reality. Likewise, mountains of data are also useless until
they can contribute to the construction of meaningful theories. Rather
than viewing these two processes in a circular relationship, as shown in
the above figure, perhaps they can be better viewed as a helix, with
each iteration between theory and data contributing to improved
observations of the phenomena and the resulting improved theory. Though
both inductive and deductive research are important for the advancement
of science, it appears that inductive (theory-building) research is more
valuable when there are few prior theories or explanations, while
deductive (theory-testing) research is more productive when there are
many competing theories of the same phenomenon and researchers are
interested in knowing which theory works best and under what
circumstances.

Theory building and theory testing are particularly difficult in
business and marketing, given the imprecise nature of the theoretical
concepts and the presence of many unaccounted factors that can influence
the phenomenon of interest. It is also very difficult to refute theories
that do not work. For instance, Karl Marx's theory of communism as an
effective economic engine withstood for decades before it was finally
discredited as being inferior to capitalism in promoting growth.
Erstwhile communist economies like the Soviet Union and China eventually
moved toward more capitalistic economies characterized by
profit-maximizing private enterprises. However, the recent collapse of
the mortgage and financial industries in the United States demonstrates
that capitalism also has its flaws and is not as effective in fostering
economic growth and social welfare as previously presumed. Unlike
theories in the natural sciences, marketing theories are rarely perfect,
which provides numerous opportunities for researchers to improve those
theories or build their own alternative theories.

Conducting scientific research, therefore, requires two sets of skills,
theoretical and methodological, needed to operate in the theoretical and
empirical levels respectively. Methodological skills (``know-how'') are
relatively standard, invariant across disciplines, and easily acquired
through various educational programs. However, theoretical skills
(``know-what'') is considerably harder to master, requiring years of
observation and reflection, and are tacit skills that cannot be taught
but rather learned though experience. All of the greatest scientists in
the history of humanity, such as Galileo, Newton, and Einstein were
master theoreticians, and they are honored for the theories they
postulated that transformed the course of science.

\hypertarget{scientific-method}{%
\subsection{Scientific Method}\label{scientific-method}}

If science is knowledge acquired through a scientific method then what
is the \emph{scientific method}? The scientific method refers to a
standardized set of techniques for building scientific knowledge, such
as how to make valid observations, how to interpret results, and how to
generalize those results. The scientific method allows researchers to
independently and impartially test preexisting theories and prior
findings, and subject them to open debate, modifications, or
enhancements. The scientific method must satisfy four characteristics:

\begin{itemize}
\item
  Replicability. Others should be able to independently replicate or
  repeat a scientific study and obtain similar, if not identical,
  results.
\item
  Precision. Theoretical concepts, which are often hard to measure, must
  be defined with such precision that others can use those definitions
  to measure those concepts and test that theory.
\item
  Falsifiability. A theory must be stated in a way that it can be
  disproven. Theories that cannot be tested or falsified are not
  scientific theories and any such knowledge is not scientific
  knowledge. A theory that is specified in imprecise terms or whose
  concepts are not accurately measurable cannot be tested, and is
  therefore not scientific. Sigmund Freud's ideas on psychoanalysis fall
  into this category and is therefore not considered a ``theory'' even
  though psychoanalysis may have practical utility in treating certain
  types of ailments.
\item
  Parsimony. When there are multiple explanations of a phenomenon,
  scientists must always accept the simplest or logically most
  economical explanation. This concept is called parsimony or ``Occam's
  razor.'' Parsimony prevents scientists from pursuing overly complex or
  outlandish theories with endless number of concepts and relationships
  that may explain a little bit of everything but nothing in particular.
\end{itemize}

Any branch of inquiry that does not allow the scientific method to test
its basic laws or theories cannot be called ``science.'' For instance,
art is not science because artistic ideas (such as the value of
perspective) cannot be tested by independent observers using a
replicable, precise, falsifiable, and parsimonious method. Similarly,
music, literature, humanities, and law are also not considered science,
even though they are creative and worthwhile endeavors.

The scientific method, as applied to business and marketing, includes a
variety of research approaches, tools, and techniques, such as
qualitative and quantitative data, statistical analysis, experiments,
field surveys, case research, and so forth. Most of this book is devoted
to learning about these different methods. However, recognize that the
scientific method operates primarily at the empirical level of research,
i.e., how to make observations and analyze and interpret these
observations. Very little of this method is directly pertinent to the
theoretical level, which is really the more challenging part of
scientific research.

Finally, business researchers must bear in mind that the natural
sciences are different from the social sciences in several important
respects. The natural sciences are very precise, accurate,
deterministic, and independent of the person making the observations.
For instance, a scientific experiment in physics, such as measuring the
speed of sound through a certain medium, should always yield the same
results, irrespective of the time or place of the experiment. However,
the same cannot be said for the social sciences, which tend to be less
accurate and more ambiguous. For instance, an economist may want to
measure the impact of some factor on a city's economy. Unfortunately,
the outcome of that research may depend on the background and experience
of the researcher, the indexes used to measure the impact, and the
interpretation of those measures. In other words, there is a high degree
variability in all social science research. While natural scientists
agree totally on the speed of light or the gravitational attraction of
the earth, there is no agreement among economists on questions like the
impact of immigration and how much of a nation's economy should be
earmarked for reducing carbon emissions. Researchers in business and
marketing must be comfortable with handling high levels of ambiguity,
uncertainty, and error that come with research in such sciences.

\hypertarget{types-of-science-research}{%
\subsection{Types of Science Research}\label{types-of-science-research}}

Depending on the purpose of research, scientific research projects can
be grouped into three types: exploratory, descriptive, and explanatory.

\hypertarget{exploratory}{%
\subsubsection{Exploratory}\label{exploratory}}

Exploratory research is often conducted in new areas of inquiry, where
the goals of the research are:

\begin{enumerate}
\def\labelenumi{\arabic{enumi}.}
\tightlist
\item
  to scope out the magnitude or extent of a particular phenomenon,
  problem, or behavior
\item
  to generate some initial ideas (or ``hunches'') about that phenomenon
\item
  to test the feasibility of undertaking a more extensive study
  regarding that phenomenon.
\end{enumerate}

For instance, if the citizens of a country are generally dissatisfied
with governmental policies during an economic recession, exploratory
research may be directed at measuring the extent of citizens'
dissatisfaction. It would consider how the dissatisfaction is manifested
and the presumed causes of such dissatisfaction. Such research may
include examination of publicly reported figures, such as estimates of
economic indicators like gross domestic product (GDP), unemployment, and
consumer price index. This research may not lead to a very accurate
understanding of the target problem, but may be worthwhile in
determining the nature and extent of the problem and serve as a useful
precursor to more in-depth research.

\hypertarget{descriptive}{%
\subsubsection{Descriptive}\label{descriptive}}

Descriptive research is directed at making careful observations and
detailed documentation of a phenomenon of interest. These observations
must be based on the scientific method and therefore, are more reliable
than casual observations by untrained people. Examples of descriptive
research are tabulation of demographic statistics by the United States
Census Bureau who use validated instruments for estimating factors like
employment by sector. If any changes are made to the measuring
instruments, estimates are provided with and without the changed
instrumentation to allow the readers to make a fair before-and-after
comparison regarding population or employment trends. Other descriptive
research may include projects like chronicling reports of gang
activities among adolescent youth, the persistence of religious,
cultural, or ethnic practices in select communities, and the role of
technologies in the spread of democracy movements.

\hypertarget{explanatory}{%
\subsubsection{Explanatory}\label{explanatory}}

Explanatory research seeks explanations of observed phenomena, problems,
or behaviors. While descriptive research examines what, where, and when
of a phenomenon, explanatory research seeks answers to why and how. It
attempts to ``connect the dots'' in research, by identifying causal
factors and outcomes of the target phenomenon. An example is
understanding the reasons behind gang violence with the goal of
prescribing strategies to overcome such societal ailments. Most academic
or doctoral research belongs to the explanation category, though some
amount of exploratory and/or descriptive research may also be needed
during initial phases of a research project. Seeking explanations for
observed events requires strong theoretical and interpretation skills,
along with intuition, insights, and personal experience.

\hypertarget{specific-considerations-for-businessmarketing-research}{%
\subsection{Specific Considerations for Business/Marketing
Research}\label{specific-considerations-for-businessmarketing-research}}

It is important to keep in mind that business researchers attempt to
explain patterns in the habits of customers. A pattern does not explain
every single person's experience, a fact that is both fascinating and
frustrating. Individuals who create a pattern may not be the same over
time and may not know one another, but they collectively create a
pattern. Those new to business research may find these patterns
frustrating because they expect various patterns to describe a group's
characteristic but that often does not translate into an actual
experience. A pattern can exist among a cohort without a specific
individual being 100\% true to that pattern.

As an example of patterns and their exceptions, consider the impact of
social class on peoples' educational attainment. In fact, Ellwood \&
Kane \footnote{Ellwood, David, and Thomas J. Kane. ``Who is getting a
  college education? Family background and the growing gaps in
  enrollment.'' Securing the future: Investing in children from birth to
  college (2000): 283-324.} found that the percentage of children who
did not receive any postsecondary schooling was four times greater among
those in the lowest quartile income bracket than those in the upper
quartile (that is, children from high-income families were far more
likely than low-income children to go to college). These research
findings detected patterns in society, but there are certainly many
exceptions. Just because a child grows up in a household with little
wealth does not keep that child from pursuing a college degree. People
who object to research findings tend to cite evidence from their own
personal experience, insisting that no patterns actually exists. The
problem with this response, however, is that objecting to a social
pattern on the grounds that it does not match a specific person's
experience misses the point about patterns.

Another matter that social scientists must consider is where they stand
on the value of basic as opposed to applied research. In essence, this
has to do with questions of for whom and for what purpose research is
conducted. We can think of basic and applied research as resting on
either end of a continuum. In marketing, basic research studies
marketing for marketing's sake --- nothing more, nothing less. Sometimes
researchers are motivated to conduct research simply because they happen
to be interested in a topic and the goal may be to learn more about a
topic. Applied research lies at the other end of the continuum. In
marketing, applied research studies marketing for some purpose beyond a
researcher's interest in a topic. Applied research is often client
focused, meaning that the researcher is investigating a question posed
by someone other than her or himself.

One final consideration for business and marketing researchers is the
difference between qualitative and quantitative methods. Qualitative
methods generally involve words (like letters, memos, or policies) or
pictures and common methods used include field research, interviews, and
focus groups. Quantitative methods, on the other hand, generally involve
numbers and common methods include surveys, content analysis, and
experimentation. While qualitative methods aim to gain an in-depth
understanding of a relatively small number of cases, quantitative
methods offer less depth but more breadth because they typically focus
on a much larger number of cases.

Sometimes these two methods are presented or discussed in a way that
suggests they are somehow in opposition to one another. The
qualitative/quantitative debate is fueled by researchers who may prefer
one approach over another, either because their own research questions
are better suited to one particular approach or because they happened to
have been trained in one specific method. While these two methodological
approaches differ in goals, strengths, and weaknesses, they both attempt
to answer a researcher's question and are equally viable. This text
operates from the perspective that qualitative and quantitative methods
are complementary rather than competing and both will be covered.

\hypertarget{summary}{%
\section{Summary}\label{summary}}

\begin{enumerate}
\def\labelenumi{\arabic{enumi}.}
\item
  There are many different sources of knowledge and some are more
  valuable than others for formulating theories and practices.
\item
  Science is the discipline of using formalized processes to create
  theories to explain observed phenomena.
\item
  Scientific research is a process with a goal of using reproducible
  methods to create a theory or validate the tenants of an existing
  theory.
\item
  Scientific research can be divided into three types: exploratory,
  descriptive, and expanatory.
\item
  Business research has specific considerations to meet the sometimes
  disparate objectives of theory-building and practical application.
\end{enumerate}

\hypertarget{foundations}{%
\chapter{Foundations}\label{foundations}}

\hypertarget{introduction-1}{%
\section{Introduction}\label{introduction-1}}

This chapter explores the connection between paradigms, theories, and
research methods and how the researcher's analytic perspective might
shape methodological choices.

\hypertarget{ontology-and-epistemology}{%
\subsection{Ontology and Epistemology}\label{ontology-and-epistemology}}

The principles of business research, like those of sociology and
psychology research, are founded on two major branches of philosophy:
\emph{ontology} and \emph{epistemology}. Ontology concerns the nature of
reality and the researcher's ontological position shapes the sorts of
research questions posed and how those questions are researched.
Ontology posits two fundamental positions:

\textbf{Objectivism}: Things are real and exist regardless of any sort
of social activity. This is often reflected in research about societal
organization. Objectivists take the position that people may differ in
their perception of reality but there is only one true reality and a
researcher's job is to discover that reality.

\textbf{Constructivism}: Things do not just exist apart from the society
that observes them. This is often reflected in research about culture
and its influence on human activities. Constructivists take the position
that reality is shaped individually and that a researcher's job is to
understand others' view of reality.

Like ontology, epistemology has to do with knowledge. Rather than
dealing with questions about \emph{what is}, epistemology deals with
questions of \emph{how we know}.

Four main branches of epistemology are frequently encountered in
business research, and the researcher's beliefs concerning these
branches will shape the research design.

\begin{enumerate}
\def\labelenumi{\arabic{enumi}.}
\item
  \textbf{Pragmatism} accepts both personal experience and measured data
  as sources of knowledge. These researchers will usually design applied
  research projects that use different perspectives to help answer a
  question.
\item
  \textbf{Positivism} relies only on findings gained through
  measurement. These researchers tend to focus on causality and try to
  reduce phenomena to its simplest elements.
\item
  \textbf{Realism} relies on observations rather than precise measures
  to provide credible facts and data. These researchers would use tools
  like structured interviews to gain an understanding of a phenomenon.
\item
  \textbf{Interpretivism} uses subjective explanations of social
  phenomena. These researchers use tools like ethnographic studies to
  attempt to understand an entire social structure.
\end{enumerate}

Burrell and Morgan (1979), in their seminal book \emph{Sociological
Paradigms and Organizational Analysis}, suggested that epistemology
shapes a researcher's approach to a project, e.g.~should an objective or
subjective approach be used, while ontology shapes the researcher's
interpretation of the findings, e.g.~does the world consist mostly of
social order or radical change. Using these two sets of assumptions,
Burrell and Morgan categorized research as in the figure below.

\begin{enumerate}
\def\labelenumi{\arabic{enumi}.}
\item
  \textbf{Functionalism} is the mindset adopted by researchers
  who\ldots{}

  \begin{itemize}
  \item
    \textbf{Ontology}: view the world as orderly and consisting of
    patterns of ordered events or behaviors.
  \item
    \textbf{Epistemology}: believe that the best way to study the world
    is to use an objective approach that is independent of the person
    conducting the observation by using standardized collection tools
    like surveys.
  \end{itemize}
\item
  \textbf{Interpretivism} is the mindset adopted by researchers
  who\ldots{}

  \begin{itemize}
  \item
    \textbf{Ontology}: view the world as orderly and consisting of
    patterns of ordered events or behaviors.
  \item
    \textbf{Epistemology}: believe that the best way to study the world
    is though the subjective interpretation of participants involved
    using techniques like interviewing participants and then reconciling
    differences using their own subjective perspectives.
  \end{itemize}
\item
  \textbf{Radical Structure} is the mindset adopted by researchers
  who\ldots{}

  \begin{itemize}
  \item
    \textbf{Ontology}: view the world as constantly changing, often
    radically, with few unvarying patterns or behaviors.
  \item
    \textbf{Epistemology}: believe that the best way to study the world
    is to use an objective approach that is independent of the person
    conducting the observation by using standardized collection tools
    like surveys.
  \end{itemize}
\item
  \textbf{Radical Humanism} is the mindset adopted by researchers
  who\ldots{}

  \begin{itemize}
  \item
    \textbf{Ontology}: view the world as constantly changing, often
    radically, with few unvarying patterns or behaviors.
  \item
    \textbf{Epistemology}: believe that the best way to study the world
    is though the subjective interpretation of participants involved
    using techniques like interviewing participants and then reconciling
    differences using their own subjective perspectives.
  \end{itemize}
\end{enumerate}

To date, the majority of business research has emulated the natural
sciences and adopted functionalist techniques. Thus, researchers tend to
believe that social patterns can be understood in terms of their
functional components so they study those components in detail using
objective techniques like surveys and experimental research. However, a
small but growing number of researchers are adopting interpretivist
techniques and are attempting to understand social order using
subjective tools such as interviews and ethnographic studies. Radical
structuralism and radical humanism represents a negligible proportion of
business research because researchers are primarily concerned with
understanding generalizable patterns of behavior rather than
idiosyncratic or changing events. However, social and organizational
phenomena generally consists of elements of both order and change. For
instance, organizational success depends on formalized business
processes, work procedures, and job responsibilities, while being
simultaneously constrained by a constantly changing mix of competitors,
competing products, suppliers, and customer base in the business
environment. Therefore, to obtain a holistic understanding of phenomena
like the success of some businesses and failure of others may require a
multi-modal approach.

\hypertarget{paradigms-and-theories}{%
\section{Paradigms and Theories}\label{paradigms-and-theories}}

The terms \emph{paradigm} and \emph{theory} are often used
interchangeably in business and marketing research although experts
disagree about whether these are identical or distinct concepts. This
text makes a slight distinction between the two ideas because thinking
about each concept as analytically distinct provides a useful framework
for understanding the connections between research methods and
scientific ways of thinking.

\hypertarget{paradigm}{%
\subsection{Paradigm}\label{paradigm}}

The researcher's own frames of reference, or believe systems, form a
paradigm. Thus, if a researcher is, generally, functionalist in outlook
then that would be the paradigm used to design and conduct research
projects. Paradigms are usually quite complex and include facets of
upbringing, family influence, societal norms, and many other factors.
Paradigms are often hard to recognize, because they are implicit,
assumed, and taken for granted. However, recognizing paradigms is key to
making sense of and reconciling differences in peoples' perceptions of
the same social phenomenon. For instance, why do liberals believe that
the best way to improve secondary education is to hire more teachers,
but conservatives believe that privatizing education (using such means
as school vouchers) are more effective in achieving the same goal? The
differences in two paradigms explains this conflict, liberals believe
more in labor (i.e., in having more teachers and schools) while
conservatives place more faith in competitive markets (i.e., in free
competition between schools competing for education dollars).

Paradigms are like ``colored glasses'' that govern how people structure
their thoughts about the world. As one other example, imagine that a
certain technology was successfully implemented in one organization but
failed miserably in another. A researcher using a ``rational lens'' will
look for rational explanations of the problem such as inadequate
technology or poor fit between technology and the task context where it
is being utilized. Another research looking at the same problem through
a ``social lens'' may seek out social deficiencies such as inadequate
user training or lack of management support. Yet another researcher
seeing it through a ``political lens'' will look for instances of
organizational politics that may subvert the technology implementation
process. Hence, subconscious paradigms often constrain the concepts that
researchers attempt to measure and their subsequent interpretations of
those measures. However, it is likely that all of the above paradigms
are at least partially correct and a fuller understanding of the problem
may require an application of multiple paradigms.

Two paradigms are commonly found in business research.

\begin{enumerate}
\def\labelenumi{\arabic{enumi}.}
\item
  Positivism. This is the framework that usually comes to mind when
  people think about scientific research \footnote{Positivism was also
    discussed as one of the main branches of epistemology but since it
    is so common in the research community it is also recognized as a
    paradigm.}. Positivism is guided by the principles of objectivity,
  knowability, and deductive logic The positivist framework operates
  from the assumption that society can and should be studied empirically
  and scientifically. Positivism also calls for value-free research
  where researchers attempt to abandon their own biases and values in a
  quest for objective, empirical, and knowable truth. Positivism is
  based on the works of French philosopher Auguste Comte (1798 - 1857)
  and was the dominant scientific paradigm until the mid-20th century.
  Unfortunately, positivism eventually evolved to empiricism or a blind
  faith in observed data and a rejection of any attempt to extend or
  reason beyond observable facts. Since human thoughts and emotions
  could not be directly measured, there were not considered to be
  legitimate topics for scientific research.
\item
  Postmodernism. Frustrations with the strictly empirical nature of
  positivist philosophy led to the development of postmodernism during
  the mid-late 20th century. Postmodernism argues that one can make
  reasonable inferences about a phenomenon by combining empirical
  observations with logical reasoning. Postmodernists view science as
  not certain but probabilistic (i.e., based on many contingencies), and
  often seek to explore these contingencies to understand social reality
  better. The postmodernist camp has further fragmented into
  subjectivists, who view the world as a subjective construction of our
  minds rather than as an objective reality, and critical realists, who
  believe that there is an external reality that is independent of a
  person's thinking but we can never know such reality with any degree
  of certainty.
\end{enumerate}

\hypertarget{theory}{%
\subsection{Theory}\label{theory}}

\hypertarget{definition}{%
\subsubsection{Definition}\label{definition}}

\emph{Theories} are explanations of a natural or social behavior, event,
or phenomenon. More formally, a scientific theory is a system of
constructs (concepts) and propositions (relationships between those
constructs) that collectively presents a logical, systematic, and
coherent explanation of a phenomenon of interest within some assumptions
and boundary conditions \footnote{Bacharach, Samuel B. ``Organizational
  theories: Some criteria for evaluation.'' Academy of management review
  14.4 (1989): 496-515.} It is important to note that people not
familiar with scientific research often view a theory as some sort of
speculation, a ``guess,'' and statements like ``it's only a theory'' are
common. However, a scientific theory is well-researched and based on
repeated observations of some phenomenon. As an example, plate tectonics
is a theory which indicates that the continents are slowly moving across
the earth's surface. This is a well-established theory based on research
spanning decades of observations, not just some sort of idle
speculation. A good scientific theory should be well supported using
observed facts and also have practical value. Famous organizational
research Kurt Lewin once said, ``Theory without practice is sterile;
practice without theory is blind.'' Hence, both theory and practice are
essential elements of research.

Theories should explain \emph{why} things happen rather than just
describe or predict. Note that it is possible to predict events or
behaviors using a set of predictors without necessarily explaining why
such events are taking place. For instance, market analysts predict
fluctuations in the stock market based on market announcements, earnings
reports of major companies, and new data from the Federal Reserve and
other agencies, based on previously observed correlations. Prediction
requires only correlations while explanations require causations.
Establishing causation requires three conditions:

\begin{enumerate}
\def\labelenumi{\arabic{enumi}.}
\tightlist
\item
  Correlations between two constructs
\item
  Temporal precedence (the cause must precede the effect in time)
\item
  Rejection of alternative hypotheses (through testing)
\end{enumerate}

It is also important to understand what theory is not. Theory is not
data, facts, typologies, taxonomies, or empirical findings. A collection
of facts is not a theory, just as a pile of stones is not a house.
Likewise, a collection of constructs (e.g., a typology of constructs) is
not a theory, because theories must go well beyond constructs to include
propositions, explanations, and boundary conditions. Data, facts, and
findings operate at the empirical or observational level, while theories
operate at a conceptual level and are based on logic rather than
observations.

There are many benefits to using theories in research. First, theories
provide the underlying logic explaining the occurrence of phenomena by
describing the key drivers, outcomes, and underlying processes that are
responsible for that phenomenon. Second, theories aid in sense-making by
synthesizing prior findings within a framework. Third, theories provide
guidance for future research by helping identify constructs and
relationships that are worthy of further research. Fourth, theories
contribute to the cumulative body of knowledge and bridge gaps between
other theories by reevaluating those theories in a new light.

However, theories can also have their own share of limitations. As
simplified explanations of reality, theories may not always provide
adequate explanations of the phenomena of interest. Theories are
designed to be simple and parsimonious explanations, while reality is
usually significantly more complex. Furthermore, theories may impose
blinders or limit researchers' ``range of vision,'' causing them to miss
out on important concepts that are not defined by the theory.

\hypertarget{building-blocks-of-a-theory}{%
\subsubsection{Building Blocks of a
Theory}\label{building-blocks-of-a-theory}}

David Whetten \footnote{Whetten, David A. ``What constitutes a
  theoretical contribution?.''" Academy of management review 14.4
  (1989): 490-495.} suggests that there are four building blocks of a
theory:

\begin{enumerate}
\def\labelenumi{\arabic{enumi}.}
\item
  Constructs capture the ``what''" of theories (i.e., what concepts are
  important for explaining a phenomenon). They are abstract concepts
  specified at a high level of abstraction that are chosen specifically
  to explain the phenomenon of interest. Constructs may be
  unidimensional (i.e., embody a single concept), such as weight or age,
  or multi-dimensional (i.e., embody multiple underlying concepts), such
  as personality or culture. While some constructs, such as age,
  education, and firm size, are easy to understand, others, such as
  creativity, prejudice, and organizational agility, may be more complex
  and abstruse, and still others such as trust, attitude, and learning,
  may represent temporal tendencies rather than steady states.
  Nevertheless, all constructs must have clear and unambiguous
  operational definition that should specify exactly how the construct
  will be measured and at what level of analysis (individual, group,
  organizational, etc.). Measurable representations of abstract
  constructs are called variables. For instance, intelligence quotient
  (IQ score) is a variable that is purported to measure an abstract
  construct called intelligence. As noted earlier, scientific research
  proceeds along two planes: a theoretical plane and an empirical plane.
  Constructs are conceptualized at the theoretical plane, while
  variables are operationalized and measured at the empirical
  (observational) plane. Furthermore, variables may be independent,
  dependent, mediating, or moderating. The distinction between
  constructs (conceptualized at the theoretical level) and variables
  (measured at the empirical level) is shown in the following figure.
\item
  Propositions capture the ``how'' (i.e., how are these concepts related
  to each other). They are associations postulated between constructs
  based on deductive logic. Propositions are stated in declarative form
  and should ideally indicate a cause-effect relationship (e.g., if X
  occurs, then Y will follow). Note that propositions may be conjectural
  but \emph{must} be testable, and should be rejected if they are not
  supported by empirical observations. However, like constructs,
  propositions are stated at the theoretical level, and they can only be
  tested by examining the corresponding relationship between measurable
  variables of those constructs. The empirical formulation of
  propositions, stated as relationships between variables, is called
  hypotheses. The distinction between propositions (formulated at the
  theoretical level) and hypotheses (tested at the empirical level) is
  depicted in the following figure.
\item
  Logic represents the ``why'' (i.e., why are these concepts related).
  Logic provides the basis for justifying the propositions as
  postulated. Logic acts like a ``glue'' that connects the theoretical
  constructs and provides meaning and relevance to the relationships
  between these constructs. Logic also represents the ``explanation''
  that lies at the core of a theory. Without logic, propositions will be
  ad hoc, arbitrary, and meaningless, and cannot be tied into a cohesive
  ``system of propositions'' that is the heart of any theory.
\item
  Boundary Conditions/Assumptions examines the ``who, when, and where''
  (i.e., under what circumstances will these concepts and relationships
  work). All theories are constrained by assumptions about values, time,
  and space, and boundary conditions that govern where the theory can be
  applied and where it cannot be applied. For example, many economic
  theories assume that human beings are rational (or boundedly rational)
  and employ utility maximization based on cost and benefit expectations
  as a way of understand human behavior. In contrast, political science
  theories assume that people are more political than rational, and try
  to position themselves in their professional or personal environment
  in a way that maximizes their power and control over others. Given the
  nature of their underlying assumptions, economic and political
  theories are not directly comparable, and researchers should not use
  economic theories if their objective is to understand the power
  structure or its evolution in a organization. Likewise, theories may
  have implicit cultural assumptions (e.g., whether they apply to
  individualistic or collective cultures), temporal assumptions (e.g.,
  whether they apply to early stages or later stages of human behavior),
  and spatial assumptions (e.g., whether they apply to certain
  localities but not to others). If a theory is to be properly used or
  tested, all of its implicit assumptions that form the boundaries of
  that theory must be properly understood. Unfortunately, theorists
  rarely state their implicit assumptions clearly, which leads to
  frequent misapplications of theories to problem situations in
  research.
\end{enumerate}

\hypertarget{variables}{%
\subsubsection{Variables}\label{variables}}

A term frequently associated with, and sometimes used interchangeably
with, a construct is a variable. Etymologically speaking, a variable is
a quantity that can vary (e.g., from low to high, negative to positive,
etc.), in contrast to constants that do not vary (i.e., remain
constant). However, in scientific research, a variable is a measurable
representation of an abstract construct. As abstract entities,
constructs are not directly measurable, and hence, we look for proxy
measures called variables. For instance, a person's intelligence is
often measured as his or her IQ (intelligence quotient) score, which is
an index generated from an analytical and pattern-matching test
administered to people. In this case, intelligence is a construct (a
concept), and the IQ score is a variable that measures that construct.
Whether IQ scores truly measures one's intelligence is anyone's guess
(though many believe that they do), and depending on whether how well it
measures intelligence, the IQ score may be a good or a poor measure of
the intelligence construct.

Depending on their intended use, variables may be classified as

\begin{itemize}
\tightlist
\item
  Independent. Explain other variables.
\item
  Dependent. Are explained by other variables.
\item
  Moderating. Influence the relationship between independent and
  dependent variables.
\item
  Mediating. Are explained by independent variables while also explain
  dependent variables.
\item
  Control. Variables that are not pertinent to explaining a dependent
  variable so must be controlled.
\end{itemize}

To understand the differences between these different variable types,
consider the example shown in the following figure.

\%Note: nice graphic on page 21/159

If the researcher believes that intelligence influences students'
academic achievement, then a measure of intelligence such as an IQ score
would be the independent variable while a measure of academic success,
grade point average, would be the dependent variable. If it is father
believed that the effect of intelligence on academic achievement is also
dependent on the students' effort then ``effort'' becomes a moderating
variable. Incidentally, it would be reasonable to also view effort as
the independent variable and intelligence as a moderating variable.

\hypertarget{attributes-of-a-good-theory}{%
\subsubsection{Attributes of a Good
Theory}\label{attributes-of-a-good-theory}}

Theories are simplified and often partial explanations of complex social
reality. As such, there can be good explanations or poor explanations,
and consequently, there can be good theories or poor theories. How can
we evaluate the ``goodness'' of a given theory? Different criteria have
been proposed by different researchers, the more important of which are
listed below:

\begin{enumerate}
\def\labelenumi{\arabic{enumi}.}
\item
  Logical consistency: Are the theoretical constructs, propositions,
  boundary conditions, and assumptions logically consistent with each
  other? If some of these ``building blocks'' of a theory are
  inconsistent with each other (e.g., a theory assumes rationality, but
  some constructs represent non-rational concepts), then the theory is a
  poor theory.
\item
  Explanatory power: How much does a given theory explain (or predict)
  reality? Good theories obviously explain the target phenomenon better
  than rival theories, as often measured by variance explained
  (R-square) value in regression equations.
\item
  Falsifiability: British philosopher Karl Popper stated in the 1940's
  that for theories to be valid, they must be falsifiable.
  Falsifiability ensures that the theory is potentially disprovable, if
  empirical data does not match with theoretical propositions, which
  allows for their empirical testing by researchers. In other words,
  theories cannot be theories unless they can be empirically testable.
  Tautological statements, such as ``a day with high temperatures is a
  hot day'' are not empirically testable because a hot day is defined
  (and measured) as a day with high temperatures, and hence, such
  statements cannot be viewed as a theoretical proposition.
  Falsifiability requires presence of rival explanations it ensures that
  the constructs are adequately measurable, and so forth. However, note
  that saying that a theory is falsifiable is not the same as saying
  that a theory should be falsified. If a theory is indeed falsified
  based on empirical evidence, then it was probably a poor theory to
  begin with!
\item
  Parsimony: Parsimony examines how much of a phenomenon is explained
  with how few variables. The concept is attributed to 14th century
  English logician Father William of Ockham (and hence called ``Ockham's
  razor'' or ``Occam's razor''), which states that among competing
  explanations that sufficiently explain the observed evidence, the
  simplest theory (i.e., one that uses the smallest number of variables
  or makes the fewest assumptions) is the best. Explanation of a complex
  social phenomenon can always be increased by adding more and more
  constructs. However, such approach defeats the purpose of having a
  theory, which are intended to be ``simplified'' and generalizable
  explanations of reality. Parsimony relates to the degrees of freedom
  in a given theory. parsimonious theories have higher degrees of
  freedom, which allow them to be more easily generalized to other
  contexts, settings, and populations.
\end{enumerate}

\hypertarget{approaches-to-theorizing}{%
\subsubsection{Approaches to
Theorizing}\label{approaches-to-theorizing}}

How do researchers build theories? Steinfeld and Fulk \footnote{Steinfield,
  Charles W., and Janet Fulk. ``The theory imperative.'' Organizations
  and communication technology (1990): 13-25.} recommend four such
approaches. The first approach is to build theories inductively based on
observed patterns of events or behaviors. Such approach is often called
``grounded theory building,'' because the theory is grounded in
empirical observations. This technique is heavily dependent on the
observational and interpretive abilities of the researcher, and the
resulting theory may be subjective and non-confirmable. Furthermore,
observing certain patterns of events will not necessarily make a theory,
unless the researcher is able to provide consistent explanations for the
observed patterns.

The second approach to theory building is to conduct a bottom-up
conceptual analysis to identify different sets of predictors relevant to
the phenomenon of interest using a predefined framework. One such
framework may be a simple input-process-output framework, where the
researcher may look for different categories of inputs, such as
individual, organizational, and/or technological factors potentially
related to the phenomenon of interest (the output), and describe the
underlying processes that link these factors to the target phenomenon.
This is also an inductive approach that relies heavily on the inductive
abilities of the researcher, and interpretation may be biased by
researcher's prior knowledge of the phenomenon being studied.

The third approach to theorizing is to extend or modify existing
theories to explain a new context, such as by extending theories of
individual learning to explain organizational learning. While making
such an extension, certain concepts, propositions, and/or boundary
conditions of the old theory may be retained and others modified to fit
the new context. This deductive approach leverages the rich inventory of
social science theories developed by prior theoreticians, and is an
efficient way of building new theories by building on existing ones.

The fourth approach is to apply existing theories in entirely new
contexts by drawing upon the structural similarities between the two
contexts. This approach relies on reasoning by analogy, and is probably
the most creative way of theorizing using a deductive approach. For
instance, Markus \footnote{Markus, M. Lynne. ``Toward a `critical mass'
  theory of interactive media: Universal access, interdependence and
  diffusion.'' Communication research 14.5 (1987): 491-511.} used
analogic similarities between a nuclear explosion and uncontrolled
growth of networks or network-based businesses to propose a critical
mass theory of network growth. Just as a nuclear explosion requires a
critical mass of radioactive material to sustain a nuclear explosion,
Markus suggested that a network requires a critical mass of users to
sustain its growth, and without such critical mass, users may leave the
network, causing an eventual demise of the network.

\hypertarget{propositions-hypothesis-and-models}{%
\section{Propositions, Hypothesis, and
Models}\label{propositions-hypothesis-and-models}}

In seeking explanations to an observed phenomenon it is not adequate
just to identify key constructs underlying that phenomenon, it is
important to also state the patterns of the relationships between
constructs. Such patterns of relationships are called propositions. A
proposition, thus, is a conjectural relationship between constructs that
is stated in a declarative form. An example of a proposition is: ``An
increase in student intelligence leads to an increase in academic
achievement.'' A proposition does not need to be true but it must be
testable so its truth can be determined. Propositions are generally
derived from either logic (deduction) or observation (induction).

Because propositions are associations between abstract constructs, they
cannot be tested directly. Instead, they are tested indirectly by
examining the relationship between the corresponding measures
(variables) of those constructs. The formulation of a proposition is
called a hypotheses. Since IQ scores and grade point average are
operational measures of intelligence and academic achievement
respectively, the proposition stated above can be specified in form of a
hypothesis: ``An increase in students' IQ score leads to an increase in
their grade point average.'' Propositions are generated from theory
while hypotheses are generated from empirical evidence. Hence,
hypotheses are testable using observed data and may be rejected if the
data do not support them.

Hypotheses are said to be either strong or weak. ``Students' IQ scores
are related to their academic achievement'' is an example of a weak
hypothesis since it indicates neither the directionality of the
hypothesis (i.e., whether the relationship is positive or negative) nor
its causality (i.e., whether intelligence causes academic achievement or
academic achievement causes intelligence). A stronger hypothesis is
``students' IQ scores are positively related to their academic
achievement,'' which indicates the directionality but not the causality.
A still better hypothesis is ``students' IQ scores have positive effects
on their academic achievement,'' which specifies both the directionality
and the causality (i.e., intelligence causes academic achievement, and
not the reverse).

Also note that hypotheses should clearly specify independent and
dependent variables. In the hypothesis, ``students' IQ scores have
positive effects on their academic achievement,'' it is clear that IQ
scores are the independent variable (the ``cause'') and academic
achievement is the dependent variable (the ``effect''). Further, it is
also clear that this hypothesis can be evaluated as either true (if
higher intelligence leads to higher academic achievement) or false (if
higher intelligence has no effect on or leads to lower academic
achievement). Statements such as ``students are generally intelligent''
or ``all students can achieve academic success'' are not hypotheses
because they do not specify independent and dependent variables nor do
they specify a directional relationship that can be evaluated as true or
false.

\hypertarget{theories-and-models}{%
\subsection{Theories and Models}\label{theories-and-models}}

A term often used in conjunction with theory is ``model.'' A model is a
representation of all or part of a system that is constructed to study
that system (e.g., how the system works or what triggers the system).
While a theory tries to explain a phenomenon, a model tries to represent
a phenomenon in an understandable way. Models are often used to make
important decisions that are based on a given set of inputs. For
instance, marketing managers may use models to decide how much money to
spend on advertising for different product lines based on parameters
such as prior year's advertising expenses, sales, market growth, and
competing products. Likewise, weather forecasters use models to predict
future weather patterns based on parameters such as wind speeds, wind
direction, temperature, and humidity. While these models are useful,
they do not explain the theory behind advertising budgets or weather
forecasting.

Models may be of different kinds, such as mathematical models, network
models, and path models. Models can also be descriptive, predictive, or
normative. Descriptive models are frequently used for representing
complex systems, for visualizing variables and relationships in such
systems. Predictive models (e.g., a regression model) allow forecast of
future events. Normative models are used to guide activities along
commonly accepted norms or practices. Models may also be static if it
represents the state of a system at one point in time or dynamic if it
represents a system's evolution over time.

The process of theory or model development may involve both inductive
and deductive reasoning, as described in Chapter 1 and shown in the
following figure. Induction occurs when we observe a fact and ask, ``Why
is this happening?'' while deduction occurs when we have a theory and
ask, ``Is this supported by observable facts?'' Both induction and
deduction leads to preliminary conclusions that are then tested in order
to develop a final model of the phenomenon. Researchers must be able to
move back and forth between inductive and deductive reasoning if they
are to post extensions or modifications to a given model or theory, or
built better ones, which are the essence of scientific research.

\%Note: nice graphic on Bhat p 24/159

\hypertarget{inductive-or-deductive-approaches}{%
\section{Inductive or Deductive
Approaches}\label{inductive-or-deductive-approaches}}

Theories are used to structure research at the same time that the
research structures theory. The reciprocal relationship between theory
and research becomes more evident to researchers as they determine
whether an inductive or deductive approach is best. Often, researchers
find that a single approach is not ideal and projects iterate over many
cycles of inductive/deductive approaches. It is common for a researcher
to start with an inductive approach and postulate a new theory then
switch to a deductive approach to test that theory. Later the researcher
may return to an inductive approach to expand and refine the theory,
followed by another round of deductive methods to test the new theory.

\hypertarget{inductive-approaches}{%
\subsection{Inductive Approaches}\label{inductive-approaches}}

In an inductive approach to research, a researcher begins by collecting
data that are relevant to the topic of interest. Once a substantial
amount of data have been collected, the researcher will start to look
for trends or correlations and then develop the theory that explains
those patterns. Thus, an inductive approach moves from data to theory,
or from specific instances to general explanations and is often referred
to as a ``bottom-up'' approach. The following figure broadly outlines
the process used with an inductive approach to research.

Inductive methods are commonly applied to qualitative research projects
and are frequently criticized for being too subjective. The goal is,
generally, to attempt to understand the dynamics of business practices
and use that understanding to draw general conclusions that may apply to
other businesses. The result of many qualitative research projects is
what is called \emph{Grounded Theory} \footnote{Grounded theory was
  first discussed by Glaser and Strauss in the late 1960's but has been
  discussed in many journal articles and books. See, for example,
  Strauss, Anselm, and Juliet Corbin. ``Grounded theory methodology.''
  Handbook of qualitative research 17 (1994): 273-85.} where the
researcher starts with no preconceived notions and generates a new
theory from the data analysis.

Following are three examples of inductive methods research.

\begin{enumerate}
\def\labelenumi{\arabic{enumi}.}
\item
  Bansal, Pratima, and Roth \footnote{Bansal, Pratima, and Kendall Roth.
    ``Why companies go green: A model of ecological responsiveness.''
    Academy of management journal 43.4 (2000): 717-736.} conducted a
  study concerning why corporations ``go green.'' They collected data
  from 53 firms in the United Kingdom and Japan and analyzed that data
  to formulate a theory.
\item
  Sharma \footnote{Sharma, Sanjay. ``Managerial interpretations and
    organizational context as predictors of corporate choice of
    environmental strategy.'' Academy of Management journal 43.4 (2000):
    681-697.} sent surveys to 3-5 senior managers of 110 Canadian oil
  and gas companies with annual sales in excess of \$20M. The surveys
  were analyzed and the researcher concludes that managers of these
  companies must be influenced to embrace environmental issues as a
  corporate goal, but that must be done within the context of the
  corporate structure.
\item
  Sia and Gopa \footnote{Sia, Surendra Kumar, and Gopa Bhardwaj.
    ``Employees `perception of diversity climate: Role of psychological
    contract.'' Journal of Indian Academy of Applied Psychology 35
    (2009): 305-312.} used an inductive method to analyze effect on
  diversity of the ``psychological contract'' between a corporation and
  its employees. A psychological contract is described as what the
  employee ``\ldots{}believes he or she has agreed to\ldots{}'' rather
  than what is actually in the employment contract. They administered
  two different surveys to 207 managers of public sector units in
  Orrisa, India. They found that certain minority groups tended to
  ``\ldots{}protect each other when required, particularly during the
  time of crisis.'' However, members of the dominant group did not
  engage in that type of behavior, leading to ``\ldots{}a feeling of
  non-inclusiveness.''
\end{enumerate}

In addition to the research studies discussed above, several papers have
been recently published by various journals encouraging inductive
research methods, especially in analyzing case studies. For example,
Eisenhardt and Graebner \footnote{Eisenhardt, Kathleen M., and Melissa
  E. Graebner. ``Theory building from cases: Opportunities and
  challenges.'' Academy of management journal 50.1 (2007): 25-32.}
published an article in the \emph{Academy of Management Journal} that
suggested a process for generating theory from multiple case studies and
encouraged management researchers to consider the role of
theory-generation in their case studies.

\hypertarget{deductive-approaches}{%
\subsection{Deductive Approaches}\label{deductive-approaches}}

In a deductive approach to research, a researcher begins with a theory
of interest and then collects data to test that theory. Thus, a
deductive approach moves from a general explanation of some phenomenon
to specific instances that prove, or disprove, the phenomenon and is
often referred to as a ``top-down'' approach. The following figure
broadly outlines the process used with a deductive approach to research.

Deductive methods are commonly applied to quantitative research projects
and are often considered the ``gold standard'' of methods, especially
among researchers in the natural sciences. The goal is, generally, to
test existing theories to see if they are valid in cases that have not
been previously considered. Following are a few example studies that use
a deductive approach.

\begin{enumerate}
\def\labelenumi{\arabic{enumi}.}
\item
  Parboteeah, Paik, and Cullen \footnote{Parboteeah, K. Praveen, Yongsun
    Paik, and John B. Cullen. ``Religious groups and work values: A
    focus on Buddhism, Christianity, Hinduism, and Islam.''
    International Journal of Cross Cultural Management 9.1 (2009):
    51-67.} studied the influence of religion on the workplace. They
  used data from more then 44 thousand individuals in 39 countries to
  determine if Buddhism, Christianity, Hindusim, and Islam influenced
  both extrinsic and intrinsic work values. They found that the results
  ``\ldots{}generally support the posited hypotheses, confirming that
  religion is positively related to work values.'' Because the study
  began with hypotheses and tested those hypotheses against gathered
  data this is a deductive methodology.
\item
  Hackman and Oldham \footnote{Hackman, J. Richard, and Greg R. Oldham.
    ``Motivation through the design of work: Test of a theory.''
    Organizational behavior and human performance 16.2 (1976): 250-279.}
  used existing theory to develop a model to predict the conditions that
  will motivate employees to perform effectively on their jobs. They
  tested 658 employees who worked at 62 different jobs in seven
  organizations and found that the results support the validity of their
  model.
\item
  Delaney and Huselid \footnote{Delaney, John T., and Mark A. Huselid.
    ``The impact of human resource management practices on perceptions
    of organizational performance.'' Academy of Management journal 39.4
    (1996): 949-969.} investigated the relationship between human
  resource management and perceptions of organizational performance.
  They came up with two hypotheses and then gathered data to test those
  hypotheses. The result of their study is that positive human resources
  practices (like training programs) have a positive correlation with
  the perception of the organizational performance.
\end{enumerate}

\hypertarget{complementary-approaches}{%
\subsection{Complementary Approaches}\label{complementary-approaches}}

While inductive and deductive approaches to research seem quite
different, they are complementary in the sense that one approach creates
theories and the other tests theories. In some cases, researchers plan
for their research to include multiple phases, one inductive and the
other deductive. In other cases, a researcher might begin a study with
the plan to only conduct either inductive or deductive research but then
discover that the other approach is needed to develop a full picture.

One such example is a research project completed by Lawrence Sherman and
Richard Berk \footnote{Sherman, Lawrence W., and Richard A. Berk. ``The
  specific deterrent effects of arrest for domestic assault.'' American
  sociological review (1984): 261-272.}. They conducted an experiment to
test two competing theories of the effects of punishment on deterring
domestic violence. Specifically, Sherman and Berk hypothesized that
deterrence theory would provide a better explanation of the effects of
arresting accused batterers than labeling theory. Deterrence theory
predicts that arresting an accused spouse batterer will reduce future
incidents of violence while labeling theory predicts that arresting
accused spouse batterers will increase future incidents.

Sherman and Berk found, after conducting an experiment with the help of
local police in one city, that arrest did in fact deter future incidents
of violence, thus supporting their hypothesis that deterrence theory
would better predict the effect of arrest. After conducting this
research, they and other researchers went on to conduct similar
experiments in six additional cities but the results from these
follow-up studies were mixed. In some cases, arrest deterred future
incidents of violence but in other cases, it did not. This left the
researchers with new data that they needed to explain. The researchers
next took an inductive approach in an effort to make sense of their
latest empirical observations. The new studies revealed that arrest
seemed to have a deterrent effect for those who were married and
employed but that it led to increased offenses for those who were
unmarried and unemployed. In the end, the researchers turned to control
theory and predicted that having some stake in conformity through the
social ties provided by marriage and employment would deter future
violence.

\hypertarget{qualitative-vs.quantitative-data}{%
\section{Qualitative vs.~Quantitative
Data}\label{qualitative-vs.quantitative-data}}

\hypertarget{summary-1}{%
\section{Summary}\label{summary-1}}

Theories, paradigms, levels of analysis, and the order in which one
proceeds in the research process all play an important role in shaping
what we ask about the business, how we ask it, and in some cases, even
what we are likely to find. A microlevel study of employment practices
will look much different than a macrolevel study. A researcher's
theoretical perspective will also shape a study. In particular, the
theory invoked will likely shape not only the way a question about a
topic is asked but also which topic gets investigated in the first
place.

This does not mean that business research is biased or corrupt. One of
the main preoccupations of researchers is to recognize and address the
biases that creep into the research process. It is human nature to
prefer a particular approach to a research project but understanding the
strengths and weaknesses of that approach is crucial for not only
successfully completing a research-based investigation but also for
intelligently reading and understanding research reports.

\hypertarget{methods}{%
\chapter{Methods}\label{methods}}

We describe our methods in this chapter.

\hypertarget{applications}{%
\chapter{Applications}\label{applications}}

Some \emph{significant} applications are demonstrated in this chapter.

\hypertarget{example-one}{%
\section{Example one}\label{example-one}}

\hypertarget{example-two}{%
\section{Example two}\label{example-two}}

\hypertarget{final-words}{%
\chapter{Final Words}\label{final-words}}

We have finished a nice book.

\bibliography{book.bib,packages.bib}


\end{document}
