%*****************************************
\chapter{Interviews}\label{ch10:interviews}
%*****************************************
%TODO Status: Pre-draft

\section{What Is Interview Research?}

\begin{wrapfigure}{r}{0.4\textwidth}
	\centering
	\includegraphics[width=0.4\textwidth]{gfx/09-men} 
\end{wrapfigure}

Today's young men are delaying their entry into adulthood. That's a nice way of saying they are ``totally confused;'' ``cannot commit to their relationships, work, or lives;'' and are ``obsessed with never wanting to grow up.''\footnote{All of the quotes in this paragraph were found at \url{http://guyland.net/}.} But don't take my word for it. Take sociologist Michael Kimmel's word. He interviewed $ 400 $ young men, ages $ 16 $ to $ 26 $, over the course of four years across the United States to learn how they made the transition from adolescence into adulthood. Since the results of Kimmel's research were published in $ 2008 $ \cite{kimmel2008perilous} his work has made quite a splash. Featured in news reports, on blogs, and in many book reviews, some claim Kimmel's research ``could save the humanity of many young men.'' Whatever is correct about Kimmel's research, one thing remains true: We surely would not know nearly as much as we now do about the lives of many young American men were it not for interview research.\footnote{Photo by Tom Cochereau on Unsplash}

Knowing how to create and conduct a good interview is an essential skill for researchers, especially those interested in \gls{qualitativeresearch}. Interviews are used by market researchers to learn how to sell their products, journalists use interviews to get information from a whole host of people from VIPs to random people on the street. Television interviewers help viewers get to know guests on their shows, employers use them to make decisions about job offers, and even radio hosts interview call-in participants. It seems everyone who's anyone knows how to conduct an interview.

From the business perspective, interviews are a method of data collection that involves two or more people exchanging information through a series of questions and answers. The questions are designed by a researcher to elicit information from interview participant(s) on a specific topic or set of topics. Typically interviews involve an in-person meeting between two people, an interviewer and an interviewee. But interviews need not be limited to two people nor must they occur in person.

Interviews are an excellent way to gather detailed information. They also have an advantage over surveys; with a survey, if a participant's response sparks some follow-up question, researchers generally do not have an opportunity to ask for more information. What they get is what they get. In an interview, however, because researchers are actually talking with the study participants in real time, they can ask follow-up questions and help clarify the responses. Thus, interviews are a useful method to find out the ``story'' behind the responses in a written survey.

Interviews are also useful when the research topic is rather complex, when the question being asked requires explanation, or when the answers to the questions may not be immediately clear to participants who may need some time in order to work through their responses. Also, if the research topic is one about which people will likely have a lot to say or will want to provide some explanation or describe some process, interviews may be the best method. 

In sum, interview research is especially useful when the following are true:

\begin{enumerate}
	\item very detailed information is requested
	\item it is anticipated that respondents will need to be asked for more information about their responses
	\item the questions require lengthy explanation
	\item the topic is complex or may be confusing to respondents
	\item the involves studying processes
\end{enumerate}

\subsection{Role of Interviewer}

The interviewer has a complex and multi-faceted role in the interview process, which includes the following tasks:

\begin{itemize}
	\item \textbf{Prepare for the interview}. Since the interviewer is in the forefront of the data collection effort, the quality of data collected depends heavily on how well the interviewer is trained to do the job. The interviewer must be trained in the interview process and the survey method, and also be familiar with the purpose of the study, how responses will be stored and used, and sources of interviewer bias. Interviewers should also rehearse and time the interview prior to the formal study.
	\item \textbf{Locate and enlist the cooperation of respondents}. Particularly in cases where the interview will take place in the participant's home, the interviewer must locate the address and work around respondents' schedule, sometimes at undesirable times such as weekends. They should also be like a salesperson, selling the idea of participating in the study.
	\item \textbf{Motivate respondents}. Respondents often feed off the motivation of the interviewer. If the interviewer is disinterested or inattentive, respondents will not be motivated to provide useful or informative responses. The interviewer must demonstrate enthusiasm about the study, communicate the importance of the research to respondents, and be attentive to respondents' needs throughout the interview. 
	\item \textbf{Clarify any confusion or concerns}. Interviewers must be able to think on their feet and address unanticipated concerns or objections raised by respondents. Additionally, they should ask probing questions as necessary even if such questions are not in the script.
	\item \textbf{Observe quality of response}. The interviewer is in the best position to judge the quality of information collected, and may supplement verbal responses obtained by also recording personal observations of gestures and body language.
\end{itemize}

\section{Qualitative Interview Techniques}

Qualitative interviews are sometimes called intensive or in-depth interviews. These interviews are semistructured; researchers have a particular topic for the interview, but questions are open ended and may not be asked in exactly the same way or in exactly the same order to each and every respondent. During in-depth interviews, the primary aim is to hear from respondents about what they think is important and to hear it in their own words. This section considers conducting qualitative interviews, analyzing interview data, and the strengths and weaknesses of this method.

\subsection{Conducting Qualitative Interviews}

Qualitative interviews might feel more like a conversation than an interview to respondents, but the researcher is in fact usually guiding the conversation with the goal of gathering information from a respondent. A key difference between qualitative and quantitative interviewing is that qualitative interviews contain open-ended questions. The meaning of this term is of course implied by its name, which are questions that a researcher poses without answer options. Open-ended questions are more demanding of participants than closed-ended questions since they require participants to come up with their own words, phrases, or sentences to respond.

In a qualitative interview, researchers usually use a guide, which is a list of topics or questions to be covered during the interview. It is called a ``guide'' because it is simply that---it is used to \textit{guide} the interview, but it is not set in stone. Think of an interview guide like an agenda for the day, it contains all the goals to be accomplished that day but it would not be the end of the world if something is skipped or if the order is changed somewhat. 

Interview guides outline issues that are important, but because participants are asked to provide answers in their own words, and to raise points that they believe are important, each interview is likely to flow a little differently. While the opening question in an in-depth interview may be the same across all interviews, from that point on what the participant says will shape how the interview proceeds. Many researchers believe that this free flow of topics makes in-depth interviewing exciting. It is also what makes in-depth interviewing rather challenging to conduct. It takes a skilled interviewer to be able to ask questions; actually listen to respondents; and pick up on cues about when to follow up, when to move on, and when to simply let the participant speak without guidance or interruption.

Interview guides tend to list topics or even specific questions, but the format of an interview guide might depend on the researcher's style, experience, and comfort level as an interviewer or with the topic. For example, in interviews of young people about their experiences with workplace sexual harassment, the guide may be topic based with few specific questions contained in the guide. Instead, it could contain only an outline of topics that are important for the research, listed in an order that it might make sense to cover them, noted on a sheet of paper.

Of course, interview guides do not appear out of thin air. They are the result of thoughtful and careful work on the part of a researcher. The topics and questions are organized thematically and in the order in which they are likely to proceed, though the flow of a qualitative interview is in part determined by what a respondent has to say. Sometimes qualitative interviewers may create two versions of the interview guide: one version contains a very brief outline of the interview, perhaps with just topic headings, and another version contains detailed questions underneath each topic heading. In this case, the researcher might use the very detailed guide to prepare and practice in advance of actually conducting interviews and then just bring the brief outline to the interview. Bringing an outline, as opposed to a very long list of detailed questions, to an interview encourages the researcher to actually listen to what a participant is saying. An overly detailed interview guide will be difficult to navigate through during an interview and could give respondents the misimpression that the interviewer is more interested in the questions than in the participant's answers.

Brainstorming is a good first step in constructing an interview guide. There are no rules at the brainstorming stage---simply list all the topics and questions that come to mind when thinking about the research question. Once a good list is created, it can be pared down by cutting questions and topics that seem redundant and grouping like questions and topics together. It is at this point that headings for grouped categories are developed. Another important avenue of approach is to consult scholarly literature to find out what kinds of questions other interviewers have asked in similar studies. As with quantitative survey research, it is best not to place very sensitive or potentially controversial questions at the very beginning of the qualitative interview guide. Participants need the opportunity to warm up to the interview and to feel comfortable talking with the interviewer. Finally, it is important to get feedback on the interview guide as it is being developed. Researchers should ask peers for guidance and suggestions once they come up with what they think is a pretty strong guide. Chances are that peer reviewers will find ways to improve the guide.

There are a few guidelines worth noting about the specific questions in the guide.

\begin{itemize}
	\item Avoid questions that can be answered with a simple yes or no. 
	\item If yes/no questions must be asked, include follow-up questions. One of the benefits of qualitative interviews is that participants can be asked for more information.
	\item While follow-up questions are appropriate, ``why'' should be avoided since this particular question can be construed as confrontational. Instead of ``why,'' something like, ``Could you tell me a little more about that?'' is a good option.
	\item Leading questions should be avoided. For example, rather than asking, ``Don't you agree that people who spend money frivolously are selfish?'' ask ``What comes to mind when you hear that someone has spent money frivolously?''
	\item Keep most, if not all, questions open ended. The key to a successful qualitative interview is giving participants the opportunity to share information in their own words and in their own way.
\end{itemize}

After the interview guide is constructed, the interviewer is still not ready to begin conducting interviews. The researcher next has to decide how to collect and maintain the information that is provided by participants. 

It is probably most common for qualitative interviewers to take audio recordings of the interviews they conduct. Recording interviews allows researchers to focus on their interaction with the interview participant rather than being distracted by trying to take notes. Of course, not all participants will feel comfortable being recorded and sometimes even the interviewer may feel that the subject is so sensitive that recording would be inappropriate. If this is the case, it is up to the researcher to balance note-taking with listening. 

Practicing the interview in advance is crucial. Ideally, researchers should interview one or two peers, or even friends, who are willing to participate in trial runs. Even better are a few people who are similar in at least some ways to the sample. The trial runs can provide feedback on the questions and the demeanor of the interviewer.

\subsection{Analysis of Qualitative Interview Data}

Analysis of qualitative interview data typically begins with a set of transcripts of the interviews. Ideally, researchers who recorded the interview can have the recordings transcribed so a written verbatim record is available. Interviewers who relied on notes taken during the interview should write a full version of the notes as quickly as possible after the interview while the session is still fresh in mind. It is usually helpful to also note non-verbal items such as body language, tone of voice, or unusually long pauses before an answer.  

While third party transcribers are easily found, it may be best for the interviewer to transcribe the recordings personally. Often, things can be recalled and noted about nonverbal behaviors and interactions that may be relevant to analysis but that could not be picked up by the audio recording alone. For example, interviewees may roll their eyes, wipe tears from their face, and even make obscene gestures that speak volumes about their feelings but would have been lost if the interviewer had not transcribed the recording personally.

The goal of analysis is to reach some inferences, lessons, or conclusions by condensing large amounts of data into relatively smaller, more manageable bits of understandable information. Analysis of qualitative interview data is normally \gls{inductiveresearch} and moves from the specific observations an interviewer collects to identifying patterns across those observations. Qualitative interviewers typically begin by reading through transcripts of their interviews and identifying codes, which is a shorthand representation of some  complex set of issues or ideas. This phase of the research is often referred to as coding and it involves reading and rereading (and rereading again) interview transcripts until the researcher has a clear idea about what sorts of themes come up across the interviews.

Qualitative researcher and textbook author Kristin Esterberg \cite{esterberg2002qualitative} describes coding as a multistage process. She suggests that there are two types of coding: open coding and focused coding. To analyze qualitative interview data, researchers can begin by open coding transcripts. They read through each transcript, line by line, and make a note of whatever categories or themes emerge. At this stage, it is important that they not let the original research question or expectations about what they think they may find cloud their ability to see new categories or themes. This is called open coding for a reason, they must keep an open mind. Open coding usually requires multiple go-rounds. As they read through the transcripts, they begin to see commonalities across the categories or themes. Then, they can begin focused coding.

Focused coding involves collapsing or narrowing themes and categories identified in open coding by reading through the notes made while conducting open coding. Researchers identify themes or categories that seem to be related, perhaps merging some or redefining others. Then they give each theme or category a name or code. Then, they identify passages of data that represent the emerging codes by reading through the transcripts yet again (and probably again). They also might write up brief definitions or descriptions of each code to making meaning of the data and develop a way to talk about the findings.

As tedious and laborious as it might seem to read through hundreds of pages of transcripts multiple times, sometimes getting started with the coding process is actually the hardest part. In their text on analyzing qualitative data, Lofland and Lofland \cite{lofland1995analytic} identify a set of questions that may be useful when coding qualitative data.

\begin{enumerate}
	\item Of what topic, unit, or aspect is this an instance?
	\item What question about a topic does this item of data suggest?
	\item What sort of answer to a question about a topic does this item of data suggest (i.e., what proposition is suggested)?
\end{enumerate}

Qualitative data can be analyzed with tools like \textit{NVivo}, \textit{RQDA}, and \textit{Coding Analysis Toolkit}\footnote{NVivo information can be found at \url{http://www.qsrinternational.com}, RQDA at \url{http://rqda.r-forge.r-project.org/}, and Coding Analysis Toolkit at \url{https://cat.texifter.com/}}. \textit{NVivo} is very powerful but expensive. \textit{RQDA} is an \textit{R} package that is useful for qualitative data analysis. Since it is part of the \textit{R} system it could be easily used in a mixed methods project where \textit{R} is used for quantitative and \textit{RQDA} is used for qualitative analysis. \textit{Coding Analysis Toolkit} is a free online text analysis service. These programs are specifically designed to assist qualitative researchers with organizing, managing, sorting, and analyzing large amounts of qualitative data. The programs work by allowing researchers to import interview transcripts and then label or code passages, cut and paste passages, search for various words or phrases, and organize complex interrelationships among passages and codes.

As an example, the following excerpt, from a paper analyzing the electronic gaming industry in two jurisdictions, \cite{buchanan2010efficacy} summarizes how the process of analyzing qualitative interview data often works.

\begin{quote}
	Data were collected through these combined methods, and while analysis was undertaken using NVivo, the analysis was guided by these methods. Thirty-eight in-depth interviews were undertaken with gaming operators and gaming machine manufacturers in both the Nevada (USA) and NSW (Australian) jurisdictions during $ 2005-2006 $. Interview data were augmented through observation, resulting in a rich collection of data. The data were coded and initially entered into ‘nodes’ within the NVivo program. A pre-defined set of themes was derived from topic areas of the interviews. Each theme then became a node. As each interview was read, additional themes were identified and nodes created for each theme. The nodes were fleshed out as data were extracted from each interview referring to the same theme. Thus a range of themes was created as a result of going through the data and coding according to themes within each transcript. Once all data had been placed into various nodes, themes were checked through the matrix function within NVivo to ensure that the various themes were distinct from each other and that there was no redundancy. 
	
	Further analysis of emerging themes resulted in a conceptual model\ldots
\end{quote}

\subsection{Strengths and Weaknesses of Qualitative Interviews}

As the preceding sections have suggested, qualitative interviews are an excellent way to gather detailed information. Whatever topic is of interest to researchers employing this method can be explored in much more depth than with almost any other method. Not only are participants given the opportunity to elaborate in a way that is not possible with other methods like survey research, but they also are able share information with researchers in their own words and from their own perspectives rather than being asked to fit those perspectives into limited response options provided by the researcher. Because qualitative interviews are designed to elicit detailed information, they are especially useful when a researcher's aim is to study processes, or the ``how'' of various phenomena. Yet another, and sometimes overlooked, benefit of qualitative interviews that it occurs in person so researchers can make observations beyond those that a respondent is orally reporting. A respondent's body language, and even her or his choice of time and location for the interview, might provide a researcher useful data.

Of course, all these benefits do not come without some drawbacks. As with quantitative survey research, qualitative interviews rely on respondents' ability to accurately and honestly recall whatever details about their lives, circumstances, thoughts, opinions, or behaviors are being asked about. Further, qualitative interviewing is time intensive and can be quite expensive. Creating an interview guide, identifying a sample, and conducting interviews are just the beginning. Transcribing interviews is labor intensive---and that's before coding even begins. It is also not uncommon to offer respondents some monetary incentive or thank-you for participating since researchers are asking for more of the participants' time than if they had simply mailed them a questionnaire. Conducting qualitative interviews is not only labor intensive but also potentially emotionally taxing. It may be that the researcher will hear stories that are shocking, infuriating, and sad.  Researchers embarking on a qualitative interview project should keep in mind their own abilities to hear stories that may be difficult to hear.

\section{Quantitative Interview Techniques}

Quantitative interviews are similar to qualitative interviews in that they involve some researcher/respondent interaction. But the process of conducting and analyzing findings from quantitative interviews differs in several ways from that of qualitative interviews. Each approach also comes with its own unique set of strengths and weaknesses.

\subsection{Conducting Quantitative Interviews}

Much of what was covered earlier in this chapter and in Chapter \ref{08:surveys}, page \pageref{08:surveys}, applies to quantitative interviews as well. In fact, quantitative interviews are sometimes referred to as survey interviews because they resemble survey-style question-and-answer formats. The difference between quantitiative interviews and surveys is that in an interview questions and answer options are read to respondents rather than having respondents complete a questionnaire on their own. As with questionnaires, the questions posed in a standardized interview tend to be closed ended. There are instances in which a quantitative interviewer might pose a few open-ended questions as well. In these cases, the coding process works somewhat differently than coding in-depth interview data.

In quantitative interviews, an interview schedule is used to guide researchers as they pose questions and answer options to respondents. An interview schedule is usually more rigid than an interview guide. It contains the list of questions and answer options that the researcher will read to respondents. Whereas qualitative researchers emphasize respondents' roles in helping to determine how an interview progresses, in a quantitative interview, consistency in the way that questions and answer options are presented is very important. The aim is to pose every question-and-answer option in the very same way to every respondent. This is done to minimize interviewer effect, or possible changes in the way an interviewee responds based on how or when questions and answer options are presented by the interviewer.

Quantitative interviews may be recorded, but because questions tend to be closed ended, taking notes during the interview is less disruptive than it can be during a qualitative interview. If a quantitative interview contains open-ended questions, however, recording the interview is advised. It may also be helpful to record quantitative interviews if a researcher wishes to assess possible interview effect. Noticeable differences in responses might be more attributable to interviewer effect than to any real respondent differences. Having a recording of the interview can help researchers make such determinations.

Quantitative interviewers are usually more concerned with gathering data from a large, representative sample but collecting data from many people via interviews can be quite laborious. Technological advances in telephone interviewing procedures can assist quantitative interviewers in this process. However, one concern about telephone interviewing is that fewer and fewer people list their telephone numbers these days, but Random Digit Dialing (\textit{RDD}) takes care of this problem. \textit{RDD} programs dial randomly generated phone numbers for researchers conducting phone interviews. This means that unlisted numbers are as likely to be included in a sample as listed numbers (though folks with unlisted numbers are not usually very pleased to receive calls from unknown researchers). Computer-assisted telephone interviewing (\textit{CATI}) programs have also been developed to assist quantitative survey researchers. These programs allow an interviewer to enter responses directly into a computer as they are provided, thus saving hours of time that would otherwise have to be spent entering data into an analysis program by hand. 

Conducting quantitative interviews over the phone does not come without some drawbacks. Responses to sensitive questions or those that respondents view as invasive are generally less accurate when data are collected over the phone as compared to when they are collected in person. Also, due to the pervasive increase in ``push polling'' for election campaigns, many respondents are unwilling to speak to a researcher on the phone.

\subsection{Analysis of Quantitative Interview Data}

As with the analysis of survey data, analysis of quantitative interview data usually involves coding response options numerically, entering numeric responses into a data analysis computer program, and then running various statistical commands to identify patterns across responses. Chapter \ref{08:surveys}, Section \ref{08:analysisOfSurveyData}, \nameref{08:analysisOfSurveyData}, page \pageref{08:analysisOfSurveyData}, describes the coding process for quantitative data. But what happens when quantitative interviews ask open-ended questions? In this case, responses are typically numerically coded, just as closed-ended questions are, but the process is a little more complex than simply giving a ``no'' a label of $ 0 $ and a ``yes'' a label of $ 1 $.

In some cases, quantitatively coding open-ended interview questions may work inductively. If this is the case, rather than ending with codes, descriptions of codes, and interview excerpts, the researcher will assign a numerical value to codes and may not utilize verbatim excerpts from interviews in later reports of results. Keep in mind that with quantitative methods the aim is to be able to represent and condense data into numbers. The quantitative coding of open-ended interview questions is often a deductive process. The researcher may begin with an idea about likely responses to his or her open-ended questions and assign a numerical value to each likely response. Then the researcher will review participants' open-ended responses and assign the numerical value that most closely matches the value of his or her expected response.

\subsection{Strengths and Weaknesses of Quantitative Interviews}

Quantitative interviews offer several benefits. The strengths and weakness of quantitative interviews tend to be couched in comparison to those of administering hard copy questionnaires. For example, response rates tend to be higher with interviews than with mailed questionnaires. That makes sense---most people find it easier to say ``no'' to a piece of paper than to a person. Quantitative interviews can also help reduce respondent confusion. If a respondent is unsure about the meaning of a question or answer option on a questionnaire, he or she probably will not have the opportunity to get clarification from the researcher. An interview, on the other hand, gives the researcher an opportunity to clarify or explain any items that may be confusing.

As with every method of data collection, there are also drawbacks to conducting quantitative interviews. Perhaps the largest, and of most concern to quantitative researchers, is interviewer effect. While questions on hard copy questionnaires may create an impression based on the way they are presented, having a person administer questions introduces a slew of additional variables that might influence a respondent. Consistency is key with quantitative data collection---and human beings are not necessarily known for their consistency. Interviewing respondents is also much more time consuming and expensive than mailing questionnaires. Thus quantitative researchers may opt for written questionnaires over interviews on the grounds that they will be able to reach a large sample at a much lower cost than were they to interact personally with each and every respondent.

\section{Issues to Consider}

While quantitative interviews resemble survey research in their question/answer formats, they share with qualitative interviews the characteristic that researchers actually interact with their subjects. The fact that researchers interacts with their subjects creates a few complexities that deserve attention.

\subsection{Power}

First and foremost, interviewers must be aware of and attentive to the power differential between themselves and interview participants. The interviewer sets the agenda and leads the conversation. While qualitative interviewers aim to allow participants to have some control over which or to what extent various topics are discussed, at the end of the day it is the researcher who is in charge (at least that is how most respondents will perceive the dynamic). Researcher are asking people to reveal things about themselves they may not typically share with others. Moreover, researchers are generally not reciprocating by revealing much or anything about themselves. All these factors shape the power dynamics of an interview.

A number of excellent pieces have been written dealing with issues of power in research and data collection. Anyan \cite{anyan2013influence} offered several suggestions for overcoming the power imbalance between researchers and participants during the data gathering phase, including the ``\ldots interviewer must court the interviewee, enhance the sense of rapport between them and build a sympathetic relationship and a sense of mutual trust in the research interview.'' During data analysis, researchers may want to consider letting interviewees interpret what they meant during the interview. ``The willingness to share the data analysis process with the interviewee or letting them join the final stages of writing is in the hands of the interviewer.'' However, researchers must be vigilant to not let the interviewee shape the outcome of the research project, researchers have an ethical obligation to maintain standards that the average interviewee would not understand.

Another easy way to balance the power differential between researchers and interview participants is to make the intent of the research project very clear. Sharing the rationale for conducting the research and the research question(s) that frame the project may help keep a proper balance of power. Participants should also understand how the data will be stored and used; and how their privacy will be protected. Many of these details are stipulated by the oversight group's procedures and requirements; but even if they are not, researchers should be attentive to how sharing information with participants can help balance the power differences between themselves and those who participate in the research project.

There are no easy answers when it comes to handling the power differential between the researcher and participants, and even professional researchers do not agree on the best approach for doing so. It is nevertheless an issue to be attentive to when conducting any form of research, particularly those that involve interpersonal interactions and relationships with research participants.

\subsection{Location}

One way to balance the power between researcher and respondent is to conduct the interview in a location of the participants' choosing, where they will feel most comfortable answering questions. Interviews can take place in any number of locations---in respondents' homes or offices, researchers' homes or offices, coffee shops, restaurants, public parks, or hotel lobbies, to name just a few possibilities. Each location comes with its own set of benefits and its own challenges. It may be best to allow the participant to choose the location that is most convenient and most comfortable, identifying a location where there will be few distractions is also important. For example, some coffee shops and restaurants are so loud that recording the interview can be a challenge. Other locations may present different sorts of distractions. For example, it may be that parents, out of necessity, will spend more time attending to their children during an interview than responding to questions. Interviewers should be prepared to suggest a few possible locations, and note the goal of avoiding distractions, when asking participants to choose a location.

Of course, the extent to which a respondent should be given complete control over choosing a location must also be balanced by accessibility of the location to the interviewer, and by the safety and comfort level of the location. It is conceivable, for example, that a participant's home could be decorated wall to wall with posters representing various white power organizations displaying a variety of violently racist messages. Even if the topic of the interview has nothing to do with home decor, the discomfort and fear the interviewer could feel during the interview could easily distracted from the task at hand. While it is important to conduct interviews in a location that is comfortable for respondents, doing so should never come at the expense of the interviewer's welfare or safety.

\subsection{Researcher-Respondent Relationship}

Finally, a unique feature of interviews is that they require social interaction, which means that to at least some extent, a relationship is formed between interviewer and interviewee. While there may be some differences in how the researcher-respondent relationship works depending on whether the interviews are qualitative or quantitative, one essential relationship element is the same: R-E-S-P-E-C-T. A good rapport between the interviewer and the participant is crucial to successful interviewing. Rapport is the sense of connection established between the interviewer and participant. Some argue that this term is too clinical and perhaps it implies that a researcher tricks a participant into thinking they are socially closer than they really are. While it is unfortunately true that some researchers might believe this implication, that is not the sense for rapport that researchers should attempt to establish with their subjects. Instead, as already mentioned, the key is \textit{respect}.

There are no big secrets or tricks for how to show respect for research participants. At its core, the interview interaction should not differ from any other social interaction in which interviewers show gratitude for a person's time and respect for a person's humanity. It is crucial that interviewers conduct the interview in a way that is culturally sensitive. In some cases, this might mean educating themselves about the study population and even receiving some training to help them learn to effectively communicate with the research participants. Interviewers should not judge participants; they are there to listen. Participants have been kind enough to give them their time and attention. Even if interviewers disagree strongly with what a participant shares in an interview, their job as researchers is to gather the information being shared, not to make personal judgments about it. 

Developing good rapport requires good listening. In fact, listening during an interview is an active, not a passive, practice. Active listening means that interviewers participate with the respondent by showing that they understand and follow whatever it is that is being shared. The questions asked to respondents should indicate that interviewers actually heard what they have just said. Active listening probably means that interviewers will probe the respondent for more information from time to time throughout the interview. A probe is a request for more information. Both qualitative and quantitative interviewers probe respondents, though the way they probe usually differs. In quantitative interviews, probing should be uniform. Often quantitative interviewers will predetermine what sorts of probes they will use. Interviewers should not to use probes that might appear to agree or disagree with what respondents said. So ``yes'' or ``I agree'' or even a questioning ``hmmmm'' should be avoided. Instead, responses like a simple ``thank you'' to indicate that the response was heard is more neutral. A ``yes'' or ``no'' response should be used if, and only if, a respondent specifically asks us if they were heard or understood.

In some ways qualitative interviews better lend themselves to following up with respondents and asking them to explain, describe, or otherwise provide more information. This is because qualitative interviewing techniques are designed to go with the flow and take whatever direction the respondent goes during the interview. Nevertheless, it is worth the interviewer's time to come up with helpful probes in advance of an interview even in the case of a qualitative interview. They do not want to find themselves stumped or speechless after a respondent has just said something about which needs to be probed further. This is another reason that practicing an interview in advance with people who are similar to those in the sample is a good idea.

\section{Conducting the Interview}

Before the interview, interviewers should prepare a ``kit'' to carry to the interview session, including a cover letter from the principal investigator or sponsor, adequate copies of the survey instrument, photo identification, and a telephone number for respondents to call to verify the interviewer's authenticity. The interviewer should set up an appointment and then be on time. To start the interview, interviewers should speak in an imperative and confident tone, such as ``I'd like to take a few minutes of your time to interview you for a very important study,'' instead of ``May I come in to do an interview?'' They should introduce themselves, present personal credentials, explain the purpose of the study in a few sentences, and assure confidentiality of respondents' comments, all in less than a minute. No big words or jargon should be used, and no details should be provided unless specifically requested. If interviewers wish to record the interview, they should ask for respondent's explicit permission before starting. Even if the interview is recorded, the interview must take notes on key issues, probes, or verbatim phrases.

During the interview, interviewers should follow the script and ask questions exactly as written. They should also not change the order of questions or skip any question that may have been answered earlier. Any issues with the questions should be discussed during rehearsal prior to the actual interview sessions. Interviewers should not finish the respondent's sentences. If the respondent gives a brief cursory answer, the interviewer should probe the respondent to elicit a more thoughtful, thorough response. Some useful probing techniques are:

\begin{itemize}
	\item \textbf{The silent probe}. Just pausing and waiting (without going into the next question) may suggest to respondents that the interviewer is waiting for more detailed response.
	\item \textbf{Overt encouragement}. Occasional ``uh-huh'' or ``okay'' may encourage the respondent to go into greater details. However, the interviewer must not express approval or disapproval of what was said by the respondent.
	\item \textbf{Ask for elaboration}. Such as ``can you elaborate on that?'' or ``A minute ago, you were talking about an experience you had in high school. Can you tell me more about that?''
	\item \textbf{Reflection}. Interviewers can try the psychotherapist's trick of repeating what the respondent said. For instance, ``What I'm hearing is that you found that experience very traumatic'' and then pause and wait for the respondent to elaborate.
\end{itemize}

After the interview in completed, interviewers should thank respondents for their time, tell them when to expect the results, and not leave hastily. Immediately after leaving, they should write down any notes or key observations that may help interpret the respondent's comments better.

\section{Focus Groups}

Focus groups resemble qualitative interviews in that a researcher may prepare an interview guide in advance and interact with participants by asking them questions. But anyone who has conducted both one-on-one interviews and focus groups knows that each is unique. In an interview, usually one member (the research participant) is most active while the other (the researcher) plays the role of listener, conversation guider, and question asker. Focus groups, on the other hand, are planned discussions designed to elicit group interaction and ``collects data through group interaction on a topic determined by the researcher.'' \cite{morgan1996focus} In this case, the researcher may play a less active role than in a one-on-one interview. The researcher's aim is to get participants talking to each other and to observe interactions among participants.

Focus groups are typically more dynamic than interviews. The researcher takes the role of moderator, posing questions or topics for discussion, but then lets the group members discuss the question or topic among themselves. Participants may ask each other follow-up questions, agree or disagree with one another, display body language that indicates something about their feelings, or even come up with questions not previously conceived of by the researcher. It is just these sorts of interactions and displays that are of interest to the researcher. A researcher conducting focus groups collects data on more than people's direct responses to her or his questions; the group interaction is a key focal point. Due to the nature and unpredictability of group interaction, and the fact that focus group researchers generally want to draw out group interaction, focus groups tend to be qualitative rather than quantitative.

There are numerous examples of marketing and business research using focus group methodology. 

In a $ 2009 $ study of the use of visual tobacco warnings, Gallopel-Morvan \etal. used focus groups and determined that the European Union graphic warnings on cigarette packages was more effective than text warnings. \cite{gallopel2011use} They used six focus group discussions conducted in Rennes, Paris, and Brest with a total of $ 50 $ people ($ 26 $ smokers, $ 24 $ non-smokers, $ 25 $ women, $ 25 $ men). 

An interesting study published by Wutich \etal in $ 2009 $ compared the results of a focus group with an open-ended self-administered questionnaire among water decision makers in Phoenix, Arizona. \cite{wutich2010comparing} She found that the focus group was no better than the questionnaires when the questions were only moderately sensitive, but the focus group was better ``\ldots for very sensitive topics when there appeared to be an opportunity to exchange important information or solve a pressing problem.''

In $ 2013 $, Sylvetsky \etal published the results of a focus group study where the effectiveness of advertising for the development of an obesity awareness campaign aimed at young people.  She conducted ten focus group discussions in two regions of Georgia, United States.  The groups of children, aged $ 9 - 14 $, were led in discussions concerning healthy food choice and lifestyles. They identified three themes: ``My Mom wants me to eat healthy foods like broccoli but it looks nasty and tastes gross,'' ``Obesity is a problem but is does not apply to me,'' and ``Everyone is made differently and it does not matter if you are fat.''

Government officials and political campaign workers use them to learn how members of the public feel about a particular issue or candidate. One of the earliest documented uses of focus groups comes from World War II when researchers used them to assess the effectiveness of troop training materials and of various propaganda efforts. \cite{merton1946focused} Market researchers quickly adopted this method of collecting data to learn about human beliefs and behaviors. Within social science, the use of focus groups did not really take off until the $ 1980 $s, when demographers and communication researchers began to appreciate their use in understanding knowledge, attitudes, and communication. Beyond various applied research projects, like those mentioned above, social scientists also use focus groups in theory development projects. 

Focus groups share many of the strengths and weaknesses of one-on-one qualitative interviews. Both methods can yield very detailed, in-depth information; are excellent for studying social processes; and provide researchers with an opportunity not only to hear what participants say but also to observe what they do in terms of their body language. Focus groups offer the added benefit of giving researchers a chance to collect data on human interaction by observing how group participants respond and react to one another. Like one-on-one qualitative interviews, focus groups can also be quite expensive and time-consuming. However, there may be some time savings with focus groups as it takes fewer group events than one-on-one interviews to gather data from the same number of people. Another potential drawback of focus groups, which is not a concern for one-on-one interviews, is that one or two participants might dominate the group, silencing other participants. Careful planning and skillful moderation on the part of the researcher are crucial for avoiding, or at least dealing with, such possibilities. The various strengths and weaknesses of focus group research are summarized below.

\begin{itemize}
	\item Strengths of Focus Group Research
	
	\begin{itemize}
		\item Yield detailed, in-depth data
		\item Less time-consuming than one-on-one interviews
		\item Useful for studying social processes
		\item Allow researchers to observe body language in addition to self-reports
		\item Allow researchers to observe interaction between multiple participants
	\end{itemize}
	
	\item Weaknesses of Focus Group Research
	
	\begin{itemize}
		\item Expensive
		\item May be more time-consuming than survey research
		\item Minority of participants may dominate entire group
	\end{itemize}
	
\end{itemize}

As mentioned, careful planning and skillful moderating are two crucial considerations in the effective use of focus groups as a method of data collection. In some ways, focus groups require more advance planning than other qualitative methods of data collection such as one-on-one interviews, where a researcher may be better able to control the setting and the dialogue, or field research, where ``going with the flow'' and observing events as they happen in their natural setting is the primary aim and time is less limited. Researchers must take care to form focus groups whose members will want to interact with one another and to control the timing of the event so that participants are not asked nor expected to stay for a longer time than they have agreed to participate. The researcher should also be prepared to inform focus group participants of their responsibility to maintain the confidentiality of what is said in the group. But while the researcher can and should encourage all focus group members to maintain confidentiality, she should also clarify to participants that the unique nature of the group setting prevents her from being able to promise that confidentiality will be maintained.

Group size should be determined in part by the topic of the interview and the researcher's sense of the likelihood that participants will have much to say without much prompting. If the topic is one about which the participants feel passionately and will have much to say, a group of three to five is ideal. Groups larger than that, especially for heated topics, can easily become unmanageable. Some recommend that a group of about six to ten participants is the ideal size for focus group research while others recommend that groups should include three to twelve participants. The size of the focus group is ultimately the researcher's decision. When forming groups and deciding how large or small to make them, researchers must take into consideration what they know about the topic and participants' potential interest in, passion for, and feelings about the topic. They must also consider their comfort level and experience in conducting focus groups. 

It may seem counterintuitive, but in general, it is better to form focus groups consisting of participants who do not know one another than to create groups consisting of friends, relatives, or acquaintances. The reason for this is that groups who know each other may share some take-for-granted knowledge or assumptions. In business research, it is precisely the taken-for-granted that is often of interest; thus the focus group researcher should avoid setting up interactions where participants may be discouraged to question or raise issues that they take for granted. However, groups should not be so heterogeneous that participants will be unlikely to feel comfortable talking with one another.

Focus group researchers must carefully consider the composition of the groups they put together. In his text on conducting focus groups, Morgan \cite{morgan1996focus} suggests that ``homogeneity in background and not homogeneity in attitudes'' should be the goal, since participants must feel comfortable speaking up but must also have enough differences to facilitate a productive discussion. Whatever composition researchers design for their focus groups, the important point to keep in mind is that focus group dynamics are shaped by multiple social contexts. Participants' silences as well as their speech may be shaped by gender, race, class, sexuality, age, or other background characteristics or social dynamics, all of which might be suppressed or exacerbated depending on the composition of the group.

In addition to the importance of advance planning, focus groups also require skillful moderation. While a researcher certainly does not want to be viewed as a stick-in-the-mud or as overly domineering, it is important to set ground rules for focus groups at the outset of the discussion. Participants should be reminded that they were invited to participate because the researcher wants to hear from all of them. Therefore, the group should aim to let just one person speak at a time and avoid letting just a couple of participants dominate the conversation. One way to do this is to begin the discussion by asking participants to briefly introduce themselves or to provide a brief response to an opening question. This will help set the tone of having all group members participate. Also ask participants to avoid having side conversations; sharing thoughts about or reactions to what is said in the group is important and should not be limited to only a few group members.

As the focus group gets rolling, the moderator will play a less active role than in a one-on-one interview. There may be times when the conversation stagnates or when the moderator wishes to guide the conversation in another direction. In these instances, it is important for moderators to demonstrate that they have been paying attention to what participants have said. Being prepared to interject statements or questions such as ``I'd really like to hear more about what Sally and Joe think about what Dominick and Ashley have been saying'' or ``Several of you have mentioned\ldots What do others think about this?'' will be important for keeping the conversation going. It can also help redirect the conversation, shift the focus to participants who have been less active in the group, and serve as a cue to those who may be dominating the conversation that it is time to allow others to speak.

In sum, focus groups are a useful method for researchers who wish to gather in-depth information about social processes. Focus groups are similar to one-on-one qualitative interviews in many ways, but they give researchers the opportunity to observe group dynamics that cannot be observed in one-on-one interviews. Historically, focus group research was more commonly used by applied researchers than by academics, though in recent decades social scientists from all domains have discovered the usefulness of focus groups for gaining understanding of social processes and have begun using this method of data collection in their studies.


\section{Summary}\label{ch10:summary}

Lorem ipsum dolor sit amet, consectetuer adipiscing elit. Aenean commodo ligula eget dolor. Aenean massa. Cum sociis natoque penatibus et
