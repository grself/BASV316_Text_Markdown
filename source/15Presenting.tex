%*****************************************
\chapter{Presenting Research}\label{ch15:writing_research}
%*****************************************
% Blackstone p 159
% Need to check Bhat and Saylor on this
%TODO Status: Pre-draft

\section{Introduction}

\section{Research as Public Activity}

In Chapter 1 "Introduction", you were introduced to the recent trend toward public sociology. As you might recall, public sociology refers to the application of sociological theories and research to matters of public interest. You might also recall that sociologists differ in their feelings about whether and the extent to which sociologists should aspire to conduct public sociology. Whether they support the movement toward public sociology or not, most sociologists who conduct research hope that their work will have relevance to others besides themselves. As such, research is in some ways a public activity. While the work may be conducted by an individual in a private setting, the knowledge gained from that work should be shared with one’s peers and other parties who may have an interest. Understanding how to share one’s work is an important aspect of the research process.

\subsection{Deciding What to Share and With Whom to Share It}

LEARNING OBJECTIVES

1. Identify the six questions social researchers should be able to answer to ensure that their ethical obligations have been met.

2. Describe how differences in one’s audience might shape how a person shares research findings.

When preparing to share our work with others we must decide what to share, with whom to share it, and in what format(s) to share it. In this section, we’ll consider the former two aspects of sharing our work. In the sections that follow, we’ll consider the various formats and mechanisms through which social scientists might share their work.

\subsubsection{Sharing It All: The Good, the Bad, and the Ugly}

Because conducting sociological research is a scholarly pursuit and because sociological researchers generally aim to reach a true understanding of social processes, it is crucial that we share all aspects of our research—the good, the bad, and the ugly. Doing so helps ensure that others will understand, be able to build from, and effectively critique our work. We considered this aspect of the research process in Chapter 3 "Research Ethics", but it is worth reviewing here.

In Chapter 3 "Research Ethics", we learned about the importance of sharing all aspects of our work for ethical reasons and for the purpose of replication. In preparing to share your work with others, and in order to meet your ethical obligations as a sociological researcher, challenge yourself to answer the following questions:

1. Why did I conduct this research?
2. How did I conduct this research?
3. For whom did I conduct this research?
4. What conclusions can I reasonably draw from this research?
5. Knowing what I know now, what would I do differently?
6. How could this research be improved?

Understanding why you conducted your research will help you be honest—with yourself and your readers—about your own personal interest, investments, or biases with respect to the work. In Chapter 4 "Beginning a Research Project", I suggested that starting where you are is a good way to begin a research project. While this is true, using the idea of starting where you are effectively requires that you be honest with yourself and your readers about where you are and why you have chosen to conduct research in a particular area. Being able to clearly communicate how you conducted your research is also important. This means being honest about your data collection methods, sample and sampling strategy, and analytic strategy.

The third question in the list is designed to help you articulate who the major stakeholders are in your research. Of course, the researcher is a stakeholder. Additional stakeholders might include funders, research participants, or others who share something in common with your research subjects (e.g., members of some community where you conducted research or members of the same social group, such as parents or athletes, upon whom you conducted your research). Professors for whom you conducted research as part of a class project might be stakeholders, as might employers for whom you conducted research. We’ll revisit the concept of stakeholders in Chapter 15 "Research Methods in the Real World".

The fourth question should help you think about the major strengths of your work. Finally, the last two questions are designed to make you think about potential weaknesses in your work and how future research might build from or improve upon your work.

\subsubsection{Knowing Your Audience}

Once you are able to articulate what to share, you must decide with whom to share it. Certainly the most obvious candidates with whom you’ll share your work are other social scientists. If you are conducting research for a class project, your main “audience” will probably be your professor. Perhaps you’ll also share your work with other students in the class. Other potential audiences include stakeholders, reporters and other media representatives, policymakers, and members of the public more generally.

While you would never alter your actual findings for different audiences, understanding who your audience is will help you frame your research in a way that is most meaningful to that audience. For example, I have shared findings from my study of older worker harassment with a variety of audiences, including students in my classes, colleagues in my own discipline (Blackstone, 2010) [1] and outside of it (Blackstone, forthcoming), [2]news reporters (Leary, 2010), [3] the organization that funded my research (Blackstone, 2008), [4] older workers themselves, and government (2010) [5] and other agencies that deal with workplace policy and worker advocacy. I shared with all these audiences what I view as the study’s three major findings: that devaluing older workers’ contributions by ignoring them or excluding them from important decisions is the most common harassment experience for people in my sample, that there were few differences between women’s and men’s experiences and their perceptions of workplace harassment, and that the most common way older workers respond when harassed is to keep it to themselves and tell no one. But how I presented these findings and the level of detail I shared about how I reached these findings varied by audience.

I shared the most detail about my research methodology, including data collection method, sampling, and analytic strategy, with colleagues and with my funding agency. In addition, the funding agency requested and received information about the exact timeline during which I collected data and any minor bureaucratic hiccups I encountered during the course of collecting data. These hiccups had no bearing on the data actually collected or relevance to my findings, but they were nevertheless details to which I felt my funder should be privy. I shared similar information with my student audience though I attempted to use less technical jargon with students than I used with colleagues.

Now that you’ve considered what to share and with whom to share it, let’s consider howsocial scientists share their research.

KEY TAKEAWAYS
• As they prepare to share their research, researchers must keep in mind their ethical obligations to their peers, their research participants, and the public.
• Audience peculiarities will shape how much and in what ways details about one’s research are reported.

EXERCISE
1. Read a scholarly article of your choice. What evidence can you find that might indicate that the author gave some thought to the six questions outlined in this section?
Saylor URL: http://www.saylor.org/books Saylor.org
161
[1] Blackstone, A. (2010, August). “The young girls thought I should be home waiting to die!” Harassment of older workers. Presentation at the Annual Meeting of the American Sociological Association, Atlanta, GA.
[2] Blackstone, A. (forthcoming). Harassment of older adults in the workplace. In P. Brownell \& J. Kelly (Eds.), Ageism in the workplace. London, UK: Springer-Verlag.
[3] Leary, M. (2010, August). Interview by Maine Public Broadcasting Network, Maine State Capitol News Service.
[4] Blackstone, A. (2008).Workplace harassment: Conceptualizations of older workers, National Science Foundation Grant SES-0817673.
[5] Blackstone, A. (2010, June).Workplace harassment: Conceptualizations and experiences of older workers in Maine. Presentation to the Maine Jobs Council, Augusta, ME.

\subsection{Presenting Your Research}

LEARNING OBJECTIVES
1. Identify the major principles of formal presentations of research.
2. Describe roundtable presentations and their benefits.
3. Discuss the purpose of and formatting principles for poster presentations.

Presenting your research is an excellent way to get feedback on your work. Professional sociologists often make presentations to their peers to prepare for more formally writing up and eventually publishing their work. Presentations might be formal talks, either as part of a panel at a professional conference or to some other group of peers or other interested parties; less formal roundtable discussions, another common professional conference format; or posters that are displayed in some specially designated area. We’ll look at all three presentation formats here.

When preparing a formal talk, it is very important to get details well in advance about how long your presentation is expected to last and whether any visual aids such as video or PowerPoint slides are expected by your audience. At conferences, the typical formal talk is usually expected to last between 15 and 20 minutes. While this may sound like a torturously lengthy amount of time, you’ll be amazed the first time you present formally by how easily time can fly. Once a researcher gets into the groove of talking about something as near and dear to him as his very own research, it is common for him to become so engrossed in it and enamored of the sound of his own voice that he forgets to watch the clock and finds himself being dragged offstage after giving only an introduction of his research method! To avoid this all-too-common occurrence, it is crucial that you repeatedly practice your presentation in advance—and time yourself.

One stumbling block in formal presentations of research work is setting up the study or problem the research addresses. Keep in mind that with limited time, audience members will be more interested to hear about your original work than to hear you cite a long list of previous studies to introduce your own research. While in scholarly written reports of your work you must discuss the studies that have come before yours, in a presentation of your work the key is to use what precious time you have to highlight your work. Whatever you do in your formal presentation, do not read your paper verbatim. Nothing will bore an audience more quickly than that. Highlight only the key points of your study. These generally include your research question, your methodological approach, your major findings, and a few final takeaways.

In less formal roundtable presentations of your work, the aim is usually to help stimulate a conversation about a topic. The time you are given to present may be slightly shorter than in a formal presentation, and you’ll also be expected to participate in the conversation that follows all presenters’ talks. Roundtables can be especially useful when your research is in the earlier stages of development. Perhaps you’ve conducted a pilot study and you’d like to talk through some of your findings and get some ideas about where to take the study next. A roundtable is an excellent place to get some suggestions and also get a preview of the objections reviewers may raise with respect to your conclusions or your approach to the work. Roundtables are also great places to network and meet other scholars who share a common interest with you.

Finally, in a poster presentation you visually represent your work. Just as you wouldn’t read a paper verbatim in a formal presentation, avoid at all costs printing and pasting your paper onto a poster board. Instead, think about how to tell the “story” of your work in graphs, charts, tables, and other images. Bulleted points are also fine, as long as the poster isn’t so wordy that it would be difficult for someone walking by very slowly to grasp your major argument and findings. Posters, like roundtables, can be quite helpful at the early stages of a research project because they are designed to encourage the audience to engage you in conversation about your research. Don’t feel that you must share every detail of your work in a poster; the point is to share highlights and then converse with your audience to get their feedback, hear their questions, and provide additional details about your research.

KEY TAKEAWAYS
• In formal presentations, include your research question, methodological approach, major findings, and a few final takeaways.
• Roundtable presentations emphasize discussion among participants.
• Poster presentations are visual representations of research findings.

EXERCISES
1. Imagine how you might present some of your work in poster format. What would the poster look like? What would it contain? Many helpful web resources offer advice on how to create a scholarly poster presentation. Simply google “scholarly poster presentation” and you’ll find hundreds of sites that share tips on creating an effective poster. Visit a few of the links that your Google search yielded. How has your vision for your poster changed and why?
2. One way to prepare yourself for presenting your work in any format is to get comfortable talking casually with others about your research. Practice with your friends and family. Engage them in a conversation about your research. Or if you haven’t conducted research yet, talk about your research interests. Ask them to repeat what they heard you express about your research project or interest. How closely do their reports match what you intended to express?

\subsection{Writing Up Research Results}

LEARNING OBJECTIVES
1. Identify the differences between reports for scholarly consumption and reports for public consumption.
2. Define plagiarism and explain why it should be taken seriously.

I once had a student who conducted research on how children interact with each other in public. She was inspired to conduct her work after reading Barrie Thorne’s (1993) [1]research on how children regulate gender through their interactions with one another. This student conducted field observations of children on playgrounds for an assignment in my research methods class. The assignment included writing up a scholarly report of findings. After writing up her scholarly report, the student revised it and submitted it for publication in the student column of Contexts, the American Sociological Association’s public-interest magazine (Yearwood, 2009). [2]Because Contexts readers run the gamut from academic sociologists to nonacademics and nonsociologists who simply have an interest in the magazine’s content, articles in the magazine are presented in a different format from the format used in other sociology journals. Thus my student had the opportunity to write up her findings in two different ways—first for scholarly consumption and then for public consumption. As she learned, and as we’ll discuss in this section, reports for fellow scholars typically differ from reports for a more general public audience.

Reports of findings that will be read by other scholars generally follow the format outlined in the discussion of reviewing the literature in Chapter 5 "Research Design". As you may recall from that chapter, most scholarly reports of research include an abstract, an introduction, a literature review, a discussion of research methodology, a presentation of findings, and some concluding remarks and discussion about implications of the work. Reports written for scholarly consumption also contain a list of references, and many include tables or charts that visually represent some component of the findings. Reading prior literature in your area of interest is an excellent way to develop an understanding of the core components of scholarly research reports and to begin to learn how to write those components yourself. There also are many excellent resources to help guide students as they prepare to write scholarly reports of research (Johnson, Rettig, Scott, \& Garrison, 2009; Sociology Writing Group, 2007; Becker, 2007; American Sociological Association, 2010). [3]

Reports written for public consumption differ from those written for scholarly consumption. As noted elsewhere in this chapter, knowing your audience is crucial when preparing a report of your research. What are they likely to want to hear about? What portions of the research do you feel are crucial to share, regardless of the audience? Answering these questions will help you determine how to shape any written reports you plan to produce. In fact, some outlets answer these questions for you, as in the case of newspaper editorials where rules of style, presentation, and length will dictate the shape of your written report.

Whoever your audience, don’t forget what it is that you are reporting: social scientific evidence. Take seriously your role as a social scientist and your place among peers in your discipline. Present your findings as clearly and as honestly as you possibly can; pay appropriate homage to the scholars who have come before you, even while you raise questions about their work; and aim to engage your readers in a discussion about your work and about avenues for further inquiry. Even if you won’t ever meet your readers face-to-face, imagine what they might ask you upon reading your report, imagine your response, and provide some of those details in your written report.

Finally, take extraordinary care not to commit plagiarism. Presenting someone else’s words or ideas as if they are your own is among the most egregious transgressions a scholar can commit. Indeed, plagiarism has ended many careers (Maffly, 2011) [4] and many students’ opportunities to pursue degrees (Go, 2008). [5] Take this very, very seriously. If you feel a little afraid and paranoid after reading this warning, consider it a good thing—and let it motivate you to take extra care to ensure that you are notplagiarizing the work of others.

KEY TAKEAWAYS
• Reports for public consumption usually contain fewer details than reports for scholarly consumption.
• Keep your role and obligations as a social scientist in mind as you write up research reports.
• Plagiarism is among the most egregious transgressions a scholar can commit.

EXERCISES
1. Imagine that you’ve been tasked with sharing the results of some of your research with your parents. What details would you be sure to include? What details might you leave out, and why?
2. Have a discussion with a few of your peers about plagiarism. How do you each define the term? What strategies do you employ to ensure that you avoid committing plagiarism?
[1] Thorne, B. (1993). Gender play: Girls and boys in school. New Brunswick, NJ: Rutgers University Press.
[2] Yearwood, E. (2009). Children and gender. Contexts, 8.
[3] Johnson, W. A., Rettig, R. P., Scott, G. M., \& Garrison, S. M. (2009). The sociology student writer’s manual (6th ed.). Upper Saddle River, NJ: Prentice Hall; Sociology Writing Group. (2007).A guide to writing sociology papers. New York, NY: Worth; Becker, H. S. (2007).Writing for social
scientists: How to start and finish your thesis, book, or article (2nd ed.). Chicago, IL: University of Chicago Press; American Sociological Association. (2010). ASA style guide (4th ed.). Washington, DC: Author. A very brief version of the ASA style guide can be found at http://www.asanet.org/students/ASA%20Quick%20Style%20Guide%204th%20edition%20update.pdf.
[4] As just a single example, take note of this story: Maffly, B. (2011, August 19). “Pattern of plagiarism” costs University of Utah scholar his job. The Salt Lake Tribune. Retrieved from http://www.sltrib.com/sltrib/cougars/52378377-78/bakhtiari-university-panel-plagiarism.html.csp?page=1
[5] As a single example (of many) of the consequences for students of committing plagiarism, see Go, A. (2008). Two students kicked off semester at sea for plagiarism. U.S. News \& World Report. Retrieved from http://www.usnews.com/education/blogs/paper-trail/2008/08/14/two-students-kicked-off-semester-at-sea-for-plagiarism

\subsection{Disseminating Findings}

LEARNING OBJECTIVES
1. Define dissemination.
2. Discuss the three considerations to keep in mind in order to successfully disseminate your findings.

Presenting your work, discussed in Section 13.2 "Presenting Your Research", is one way of disseminating your research findings. In this section, we’ll focus on disseminating the written results of your research. Dissemination refers to “a planned process that involves consideration of target audiences and the settings in which research findings are to be received and, where appropriate, communicating and interacting with wider policy and…service audiences in ways that will facilitate research uptake in decision-making processes and practice” (Wilson, Petticrew, Calnan, \& Natareth, 2010, p. 91). [1] In other words, dissemination of research findings involves careful planning, thought, consideration of target audiences, and communication with those audiences. Writing up results from your research and having others take notice are two entirely different propositions. In fact, the general rule of thumb is that people will not take notice unless you help and encourage them to do so. To paraphrase the classic line from the film Field of Dreams, just because you build it doesn’t mean they will come.

Disseminating your findings successfully requires determining who your audience is,where your audience is, and how to reach them. When considering who your audience is, think about who is likely to take interest in your work. Your audience might include those who do not express enthusiastic interest but might nevertheless benefit from an awareness of your research. Your research participants and those who share some characteristics in common with your participants are likely to have some interest in what you’ve discovered in the course of your research. Other scholars who study similar topics are another obvious audience for your work. Perhaps there are policymakers who should take note of your work. Organizations that do work in an area related to the topic of your research are another possibility. Finally, any and all inquisitive and engaged members of the public represent a possible audience for your work.

Where your audience is should be fairly obvious once you’ve determined who you’d like your audience to be. You know where your research participants are because you’ve studied them. You can find interested scholars on your campus (e.g., perhaps you could offer to present your findings at some campus event), at professional conferences, and via publications such as professional organizations’ newsletters (an often-overlooked source for sharing findings in brief form) and scholarly journals. Policymakers include your state and federal representatives who, at least in theory, should be available to hear a constituent speak on matters of policy interest. Perhaps you’re already aware of organizations that do work in an area related to your research topic, but if not, a simple web search should help you identify possible organizational audiences for your work. Disseminating your findings to the public more generally could take any number of forms: a letter to the editor of the local newspaper, a blog, or even a post or two on your Facebook wall.

Finally, determining how to reach your audiences will vary according to which audience you wish to reach. Your strategy should be determined by the norms of the audience. For example, scholarly journals provide author submission instructions that clearly define requirements for anyone wishing to disseminate their work via a particular journal. The same is true for newspaper editorials; check your newspaper’s website for details about how to format and submit letters to the editor. If you wish to reach out to your political representatives, a call to their offices or, again, a simple web search should tell you how to do that.

Whether you act on all these suggestions is ultimately your decision. But if you’ve conducted high-quality research and you have findings that are likely to be of interest to any constituents besides yourself, I would argue that it is your duty as a scholar and a sociologist to share those findings. In sum, disseminating findings involves the following three steps:

1. Determine who your audience is.
2. Identify where your audience is.
3. Discover how best to reach them.

KEY TAKEAWAYS
• Disseminating findings takes planning and careful consideration of one’s audiences.
• The dissemination process includes determining the who, where, and how of reaching one’s audiences.

EXERCISES
1. What additional potential audiences for your research, aside from those already mentioned, can you identify? How might you reach those audiences?
2. Do you agree or disagree with the assertion that researchers who conduct high-quality research have a duty to share their findings with others? Explain.
[1] Wilson, P. M., Petticrew, M., Calnan, M. W., \& Natareth, I. (2010). Disseminating research findings: What should researchers do? A systematic scoping review of conceptual frameworks.Implementation Science, 5, 91.


\section{Summary}\label{ch04:summary}

Lorem ipsum dolor sit amet, consectetuer adipiscing elit. Aenean commodo ligula eget dolor. Aenean massa. Cum sociis natoque penatibus et
