%*****************************************
\chapter{Presenting Research}
%*****************************************
% Blackstone p 159
% Need to check Bhat and Saylor on this
%TODO Status: Pre-draft

\section{Introduction}

Most researchers hope that their work will have relevance to others besides themselves. As such, research is in some ways a public activity. While the work may be conducted by an individual in a private setting, the knowledge gained from that work should be shared with  peers and other parties who may have an interest. Understanding how to share research is an important aspect of the research process.

\section{What and With Whom to Share}

When preparing to share work with others, researchers must decide what to share, with whom to share it, and in what format(s) to share it. This section considers the ``what'' and ``with whom'' aspects while later sections cover the various formats and mechanisms through which research is shared.

\subsubsection{Sharing It All}

Because conducting research is a scholarly pursuit and because researchers generally aim to reach a true understanding of business and economics processes, it is crucial that all aspects of research, the good, the bad, and the ugly, are shared. Doing so helps ensure that others will understand, be able to build from, and effectively critique the work.

It is important to share all aspects of a research project for ethical reasons and to permit other researchers to replicate the work. The following questions will aid researchers in preparing to share research with others.

\begin{enumerate}
	\item Why was the research conducted?
	\item How was the research conducted?
	\item For whom was the research conducted?
	\item What conclusions can be reasonably drawn from this research?
	\item How could the research have been improved?
\end{enumerate}

Answering these questions help researchers be honest with themselves and the readers about their own personal interest, investments, or biases with respect to the work. The third question helps identify the major stakeholders, like funders, research participants, or others who share something in common with the research project (e.g., members of the community or social group who were involved in the research). These groups may be interested in the outcome of the research but may also be a source of bias. The last two questions help identify the strengths and weaknesses of the research project and could point the way for future projects.

%TODO Start Here
\subsubsection{Knowing Your Audience}

An important decision for researchers is determining with whom to share the results. Certainly, the most obvious candidates with are other researchers working in the same field. Other potential audiences include stakeholders, reporters and other media representatives, policymakers, and members of the public more generally.

While the findings of a research project would never be altered for different audiences, understanding the audience helps frame the research report in a way that is most meaningful to that group. For example, the report for a project about the spending habits of elderly pensioners may be much different if rendered for a group of business owners, a governmental committee on aging, the funding agency, and a community meeting. In all cases, researchers would share the study's major findings, but the method of presentation and level of detail would vary by audience.

It would be expected that the greatest amount of detail, including data collection method, sampling, and analytic strategy, would be shared with colleagues and the funding agency. In addition, the funding agency may want information about the exact time line for the project along with any bureaucratic hiccups encountered. With a community meeting, though, a more succinct summary of the important findings using less technical jargon would be appropriate.

\subsection{Oral Presentations}

Researchers frequently make presentations to their peers in settings like conferences or departmental meetings. These presentations are excellent means for feedback and help researchers prepare to write up and publish their work. Presentations might be formal talks, either as part of a panel at a professional conference or to some other group of peers or other interested parties; less formal roundtable discussions, another common professional conference format; or posters that are displayed in some specially designated area.

When preparing a formal talk, it is very important for researchers to get details well in advance about the time limit for the presentation, requirements for questions from the audience, and whether visual aids, such PowerPoint slides, are expected. At conferences, the typical formal talk is usually expected to last between 15 and 20 minutes. Once researchers start talking about something as as important as their own research, it is common for them to become so engrossed that they forget to watch the clock and finds themselves running short of time. To avoid this all-too-common occurrence, it is crucial that presenters practice in advance, and time themselves.

One common mistake made in formal presentations of research work is in setting up the problem the research addresses. Audience members are usually more interested to hear about the researcher's work work than to hear the results of a long list of previous related studies. While written reports must discuss related previous studies, presentations must use the precious time available to highlight the current research project. Another mistake is to simply read the research paper verbatim. Nothing will bore an audience more quickly than hearing a presenter drone on while reading aloud. Finally, a presentation should highlight only the key points of the study, which, generally, include the research question, methodological approach, major findings, and a few final takeaways.

In less formal round table presentations, the aim is usually to help stimulate a conversation about a topic. Normally, several research projects are presented so the time available for each is normally shorter than in a formal presentation. Also, round table presentations always includes time for a conversation following the presentations. Round tables are especially useful when a research project is in the early stages of development. For example, perhaps a researcher has conducted a pilot study and is interested in ideas about where to take the study next. A round table is also an excellent place for a preview of potential objections reviewers may raise with respect to the project's approach or conclusions. Finally, round tables are great places to network and meet other scholars who share common interests.

Finally, a poster presentation is a visual representation a research project. Often, poster sessions are tables lined up in a conference area where researchers have a visual display on the table but also stand by to answer questions. A poster should not be just pages from the report pasted onto a poster board, rather, researchers decide how to tell the ``story'' of the work in graphs, charts, tables, and other images. Bulleted points are acceptable as long as the people walking by can quickly read and grasp the major argument and findings. Posters, like round tables, can be quite helpful at the early stages of a research project because they are designed to encourage the audience to engage in conversation about the research. It is not necessary to share every detail of a research project in a poster, the point is to share highlights and then discuss the details with people who are interested.

\subsection{Written Presentations}

Written reports that will be read by other scholars generally follow a formal format that is outlined by the publication journal. However, most scholarly reports include an abstract, an introduction, a literature review, a discussion of research methodology, a presentation of findings, and some concluding remarks and discussion about implications of the work. Reports written for scholarly consumption also contain a list of references and many include tables or charts that visually represent some component of the findings. Reading published research in business or economics is an excellent way to develop an understanding of the core components of scholarly research reports and to begin to learn how to write those components.

Reports written for public consumption differ from those written for scholarly consumption. As noted elsewhere in this chapter, knowing the audience is crucial when preparing a written report. Whoever your audience, it is important to keep in mind that scientific evidence is being reported. Writers must take seriously their roles as business researchers and be mindful of their place among peers in the discipline. Findings must be presented as clearly and honestly as possible; appropriate recognition must be afforded to the scholars who have come before, even if the research raises questions about their work; and readers should be engaged in a discussion about the research and potential avenues for further inquiry. Normally, research writers will never meet the readers face-to-face, but it is beneficial to imagine what the readers would ask and provide a detailed response in the written report.

Finally, it is extremely important to not to commit plagiarism in a research report. Presenting someone else's words or ideas as if they are the researcher's own is among the most egregious transgressions a scholar can commit. Indeed, plagiarism has ended many careers (Maffly, 2011) [4] and many students' opportunities to pursue degrees (Go, 2008). [5] 


\subsection{Disseminating Findings}

This section focuses on disseminating the written results of a research project. Dissemination refers to ``a planned process that involves consideration of target audiences and the settings in which research findings are to be received and, where appropriate, communicating and interacting with wider policy and…service audiences in ways that will facilitate research uptake in decision-making processes and practice'' (Wilson, Petticrew, Calnan, \& Natareth, 2010, p. 91). [1] In other words, dissemination of research findings involves careful planning, thought, consideration of target audiences, and communication with those audiences. Writing up results from a research project and having others take notice are two entirely different propositions. In fact, the general rule of thumb is that people will not take notice unless they are encouraged to do so. To paraphrase the classic line from the film \textit{Field of Dreams}, just because you build it does not mean they will come.

Disseminating research findings successfully requires determining who the audience is, where that audience is located, and how to reach them. When considering who the audience is, think about who is likely to take interest in the research project. The audience might include those who do not express enthusiastic interest but might nevertheless benefit from an awareness of the research. Of course, the research participants and those who share some characteristics in common with those participants are likely to have some interest in what was discovered in the course of the research. Other scholars who study similar topics are another obvious audience for the work. Perhaps there are policymakers who should take note of the work. Organizations that do work in an area related to the topic of the research project are another possibility. Finally, any and all inquisitive and engaged members of the public represent a possible audience for the work.

Where the audience is located should be fairly obvious once the composition of that audience is determined. The research participants are known since they were part of the study. Interested scholars can be found at professional conferences and via publications such as professional organizations' newsletters and scholarly journals. Policymakers include state and federal representatives who, at least in theory, should be available to hear a constituent speak on matters of policy interest. Organizations that do work in an area related to the research topic can be found with a simple web search. Finally, disseminating findings to the general public could take any number of forms: a letter to the editor of a local newspaper, a blog, or even a Facebook post.

Finally, determining how to reach the target audience will vary depending on which specific audience is of interest. The strategy should be determined by the norms of the audience. For example, scholarly journals provide author submission instructions that clearly define requirements for researchers wishing to disseminate their work via that journal. The same is true for newspaper editorials; check your newspaper's website for details about how to format and submit letters to the editor. To reach out to political representatives, a call to their offices or a simple web search should information about how to proceed.

Researchers who have conducted high-quality research and have findings that are likely to be of interest to any constituents besides themselves would have a duty as a scholar to share those findings. 

\section{Summary}\label{ch15:summary}

Lorem ipsum dolor sit amet, consectetuer adipiscing elit. Aenean commodo ligula eget dolor. Aenean massa. Cum sociis natoque penatibus et
